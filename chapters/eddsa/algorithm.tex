Ed25519 における3つのアルゴリズムの手順を以下に述べる.\\[1em]
\begin{mdframed}[linecolor=black,roundcorner=10pt]
  \textbf{鍵生成}
  \begin{enumerate}[parsep=7pt]
    \item 法とする素数$p$、楕円曲線$E$、基準点$B$、
          鍵のサイズ$b$、ハッシュ関数$H$、エンコーディング関数$E'$、
          コファクター$c$、位数$L$を定める.
    \item $b$バイトのランダムな値$sk$を生成し、\textbf{秘密鍵}とする.
    \item $h = H(sk)$を計算し、$h$(オクテット文字列)を
    前半部分$h[0]$から$h[31]$と後半部分$h[32]$から$h[63]$に分ける.
    \item 前半部分の最初のバイト($h[0]$)の下位3ビットを0にクリアする.
    最後のバイト($h[31]$)の最上位ビットを0に、最上位2ビット目を1に設定したものを
    リトルエンディアンの整数として解釈し、スカラー$s$ (mod $L$)を生成する.
    \item 基準点$B$を使って$A = sB$を計算し、$ENCE(A)$を\textbf{公開鍵}とする.
  \end{enumerate}
  スカラー値$s$は 8 の倍数で,正確に 255 ビットとなる
\end{mdframed}
\vspace{2em}
\begin{mdframed}[linecolor=black,roundcorner=10pt]
  \textbf{署名生成}
  \begin{enumerate}[parsep=7pt]
    \item 秘密鍵$sk$を使って、ハッシュ値$h=H(sk)$を計算する.
    \item $h$の後半部分$h[32]$から$h[63]$を使って、
    \begin{center}
      $r = DEC(H(h[32]\mid\mid ... \mid\mid h[63] \mid\mid M))$ mod $L$.
    \end{center}
    を計算する.
    \item $R=ENCE([r]B)$ を計算する.
    \item $k=DEC(H(R \mid\mid A \mid\mid M))$ mod$L$ を計算する.
    \item $S=ENC((r+k*s)$ mod$L)$ を計算する.
    \item $(R,S)$を署名とする.
  \end{enumerate}
  秘密のスカラー値
\end{mdframed}
\vspace{2em}
\begin{mdframed}[linecolor=black,roundcorner=10pt]
  \textbf{署名検証}
  \begin{enumerate}[parsep=7pt]
    \item 署名$(R,S)$を受け取り、$R$をデコードする.
    \item
  \end{enumerate}
\end{mdframed}

