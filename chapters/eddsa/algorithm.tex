Ed25519 における3つのアルゴリズムの手順を以下に述べる.\\
\indent なお, ENC, DEC, ENCE, DECEは2.2節で述べた関数を使用する.
また, $H, L, B$は2.3節で述べたEd25519の値, または関数を使用する.\\[1em]
\let\ltxlist\list
\begin{breakitembox}[l]{\textbf{鍵生成}}
   
  \begin{enumerate}[parsep=7pt]
    \item $256$バイトのランダムな値$Pk$を生成し, 秘密鍵とする.
    \item $h=H(Pk)=(h[0]\mid\mid h[1]\mid\mid \cdots\mid\mid h[63])$
    を計算する. (ここで$h[n]$はハッシュ関数の出力512ビットを
    オクテット配列と見なしたときの$n$オクテット目のビット列を表す.)
    \item $h$を前半部分 $h[0]$ から $h[31]$ と後半部分 $h[32]$ から $h[63]$ に分ける.
    \item 前半部分の最初のバイト($h[0]$)の下位3ビットを0にクリアする.
    最後のバイト($h[31]$)の最上位ビットを0に, 最上位2ビット目を1に設定したものを
    リトルエンディアン形式の整数として解釈し, スカラー $s \mod L$を生成する.
    \item 基準点$B$を使って$A = sB$を計算し, $ENCE(A)$を公開鍵とする.
  \end{enumerate}
  スカラー値$s$は 8 の倍数で,ちょうど 255 ビットとなる.
\end{breakitembox}
\vspace{2em}
\let\ltxlist\list
\begin{breakitembox}[l]{\textbf{署名生成}}
   
  \begin{enumerate}[parsep=7pt]
    \item 秘密鍵$Pk$を使って, ハッシュ値 $h=H(Pk)$ を計算する.
    \item $h$の後半部分 $h[32]$ から $h[63]$ を使って, 
    \begin{center}
      $r \equiv H(h[32]\mid\mid \cdots \mid\mid h[63] \mid\mid M) \mod L$
    \end{center}
    を計算し, デコードする.
    \item $R=rB$ を計算し, エンコードする.
    \item $k=H(R \mid\mid A \mid\mid M)$ を計算する.
    \item 鍵生成の際 $s$ を用いて $S\equiv r+ks \pmod L$ を計算し, エンコードする.
    \item $(R,S)$を署名とする.
  \end{enumerate}
\end{breakitembox}
\vspace{2em}
\let\ltxlist\list
\begin{breakitembox}[l]{\textbf{署名検証}}
 
  \begin{enumerate}[parsep=7pt]
    \item メッセージ$M$と署名$(R,S)$を受け取る.
    \item $R'=DECE(R)$ として, $R$をデコードする.
    \item $S'=DEC(S)$ として, $S$をデコードする.
    \item $A'=DECE(A)$ として, $A$をデコードする.
    \item 2~4でデコードに失敗した場合や$0\leq S<L$の範囲外である場合, 
    署名を棄却する.
    \item $k'=H(R' \mid\mid A' \mid\mid M)$ を計算する.
    \item $8S'B=8R'+8k'A'$ が成り立てば, 署名は有効である.
    そうでなければ, 署名は無効である.
  \end{enumerate}
\end{breakitembox}

