EdDSAのパラメータを表\ref{tab:eddsa-parameters}にまとめる.\\
\captionsetup{font=normalsize} % キャプションのフォントサイズを通常にする設定
\begin{table}[htbp]
  \caption{EdDSAのパラメータ}
  \label{tab:eddsa-parameters}
  \centering
  \begin{tabular}{cp{10cm}}
    \hline
    \multicolumn{1}{c}{パラメータ} & \multicolumn{1}{c}{説明} \\ \hline \hline
    $p$ & 法となる素数. EdDSAは$\mathbb{F}_p$上のエドワーズ曲線を使用する.\\
    $b$ & $b\geq 10$かつ$p<2^{b-1}$となる正整数. 公開鍵の長さを表す.\\
    H & ハッシュ関数. 2bビット長のハッシュ値を出力する. \\
    $E$ & EdDSAで使用するエドワーズ曲線. $E$は$\mathbb{F}_p$上の楕円曲線であり, $E$は以下の式で定義される.\\
    $a$ & $E$を決定するパラメータ. $\mathbb{F}_p$上の平方剰余. すなわち, $x^2\equiv a \pmod{p}$となる$x$が存在する.\\
    $d$ & $E$を決定するパラメータ. $a\neq d$. 非ゼロの非剰余. すなわち, $x^2\equiv d \mod{p}$となる$x$が存在しない.\\
    $c$ & $E$を決定するパラメータ. $c=2$ or $3$. $2^{c}$はコファクタと呼ばれる.\\
    $L$ & $E$を決定するパラメータ. $B$の位数. $2^{200}$より大きい奇素数. $E$の有理点の個数が$\sharp E=2^{c}l$となるような数である.\\
    $B$ & $E$上のベースポイント. $B\neq (0,1)$\\
    $n$ & $c\leq n < b$となる整数.\\ \hline
  \end{tabular}
\end{table}

本研究で使用するEd25519のパラメータを表\ref{tab:ed25519-parameters}にまとめる.\\
\begin{longtable}{cl}
  \caption{Ed25519のパラメータ}
  \label{tab:ed25519-parameters}
  \endfirsthead
  \hline
  \multicolumn{1}{c}{パラメータ} & \multicolumn{1}{c}{値, または関数} \\ \hline \hline
  $p$ & $2^{255}-19$ \\
  $b$ & 256 \\
  H & SHA-512 \\
  $E$ & ツイストエドワーズ曲線 Curve25519 \\
  $a$ & $-1$ \\
  $d$ & $-\frac{121665}{121666}$ \\
  $c$ & 3 \\
  $L$ & $2^{252} + 27742317777372353535851937790883648493$ \\
  $B$ & $(15112221349535400772501151409588531511454012693$ \\
  & $041857206046113283949847762202,$\\
  & $4631683569492647816942839400347516314130799386625622$ \\
  & $5615783033603165251855960)$ \\
  $n$ & 254 \\ \hline
\end{longtable}

