\indent 階戸\cite{shinato}は, デジタル署名方式としてECDSAを使用してシミュレーションを行い, 
署名に関する計算負荷と通信への影響を評価していた. 
一方, この章では, EdDSAがECDSAよりもセキュリティと処理時間において優れた署名方式であり, 通信性能
を向上させる可能性を示してきた. そこで, 本研究では, 署名方式をEdDSAに変更することで
階戸のプロトコルの更なる効率化を図る. さらに, EdDSAを組み込んだプロトコルの作成と, 
そのプロトコルを用いたシミュレーションを実行し, EdDSAがGPSRにおいて, 
どのような影響を与えるかをECDSAと比較することで評価を行う. \\
\indent 階戸は, ns-3のGPSRのモジュールにOpen SSLのDSAとECDSAの署名機能を追加する
コーディング行っていた. 本研究では, EdDSAを導入するために, Open SSLのEd25519
\cite{openssl-eddsa}の署名機能をGPSRのモジュールに追加するようコーディングを行った. \\