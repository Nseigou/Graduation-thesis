Ed25519はECDSAと比べ, 以下の点でセキュリティと処理時間が向上している.\\[1em]
{\large\textbf{セキュリティ}}\\
\noindent\paragraph{決定論的な署名生成} \leavevmode\\
 Ed25519 は, 署名ごとにランダムな値を生成する代わりに, 
メッセージと秘密鍵をハッシュすることで 
ノンス(署名に使用するランダムな数値)を決定論的に生成する.
この方法により, 乱数生成の失敗による秘密鍵の漏洩リスクを回避している.
一方, ECDSA では, 乱数が予測可能または重複した場合, 秘密鍵が簡単に
推測されるリスクがあり, これがセキュリティの大きな弱点となる.
\vspace{1em}
\noindent{\large\textbf{処理時間}}\\[1em]
 Ed25519 は, ねじれたエドワーズ曲線を使用しており, 
この形式の曲線は楕円曲線演算を効率的に実行できる特性を持っている.
特に, 加算と倍加の操作が簡素化され, 計算ステップが少なく済むため, 
ソフトウェアでの実装が高速になる.\\

\noindent\paragraph{乱数生成の回避} \leavevmode\\
 EdDSA では乱数を鍵生成でのみ使用している.この乱数は秘密鍵に直接影響を
与えるため,暗号的に安全である必要がある.一方で,署名生成,署名検証では
乱数を使用しない.一般的に,暗号的に安全な乱数は生成の時間がランダムで
処理コストが高い.EdDSA では署名生成,署名検証を一定時間で行い,
乱数を使用するアルゴリズムに比べて高速に処理することができる.\\

\noindent\paragraph{乗法逆元の不要} \leavevmode\\
 Ed25519では拡張した射影座標系(拡張ツイストエドワーズ座標)で
計算することがRFC8032で推奨されている.
この座標系の特性により, 処理コストの高い乗法逆元の処理が不要となり
処理が単純化され, 高速化しやすい.

