EdDSAはECDSAと比べ, セキュリティと処理時間の点で性能が向上している. 
その主な3つの理由を以下に挙げる. \\[0.5em]
\noindent\textbf{(1) 決定論的な署名生成}\\
\indent Ed25519 は, 署名ごとにランダムな値を生成する代わりに, 
メッセージと秘密鍵をハッシュすることで, 
ノンス(署名に使用するランダムな数値)を決定論的に生成する.
この方法により, 乱数生成の失敗による秘密鍵の漏洩リスクを回避している. \\
\indent 一方, ECDSA では, 乱数が予測可能または重複した場合, 秘密鍵が簡単に
推測されるリスクがあり, これがセキュリティの大きな弱点となる.\\[0.5em]
\noindent\textbf{(2) 乱数生成の回避} \leavevmode\\
\indent EdDSA では乱数を鍵生成でのみ使用している.この乱数は秘密鍵に直接影響を
与えるため,暗号的に安全である必要がある.一方で,署名生成,署名検証では
乱数を使用しない.一般的に,暗号的に安全な乱数は生成の時間がランダムで
処理コストが高い. \\
\indent EdDSA では署名生成,署名検証を一定時間で行い,
乱数を使用するアルゴリズムに比べて高速に処理することができる.\\[0.5em]
\noindent\textbf{(3) 乗法逆元の不要} \leavevmode\\
\indent RFC8032において, Ed25519では拡張した射影座標系 (拡張ツイストエドワーズ座標)で
計算することが推奨されている.
この座標系の特性により, 処理コストの高いスカラー値の乗法逆元の計算が不要となる. 
そのため, 処理が単純化され, 高速化しやすい.

