この節ではEdDSAで使用される\textbf{ツイストエドワーズ曲線
(Twisted Edwards Curve)}について説明する.\\
\indent エドワーズ曲線は, 体$\mathbb{F}_p$ ($p$は素数かつ$p\neq 2$)上で定義され, 
元$c,d\in \mathbb{F}_p, cd(cd^{4}-1)\neq 0$に対し, 
以下の式で与えられる.
\begin{equation}
  x^2 + y^2 = c^2(1+dx^2y^2)
\end{equation}
この曲線は, 変数変換することでWeierstrass方程式(式1.1)で定まる楕円曲線と
同型であることが知られている\cite{edwarscurve-to-ellipticcurve}.\\
\indent ツイストエドワーズ曲線は体$\mathbb{F}_p$ ($p$は素数かつ$p\neq 2$)上で定義され, 
異なる非零元$a,d\in \mathbb{F}_p$ に対し, 以下の式で与えられる.
\begin{equation}
  ax^2 + y^2 = 1 + dx^2y^2
\end{equation}
ツイストエドワーズ曲線は, 通常のエドワーズ曲線(式2.1)と
拡大体$\mathbb{F}_p (\sqrt{a})$上で同型である. この同型により, 
ツイストエドワーズ曲線上の点集合は群をなす\cite{twisted}.\\[1em]
\noindent{\large\textbf{ツイストエドワーズ曲線上の楕円加算}}\\
\indent ツイストエドワーズ曲線上の点$(x_1,y_1)$と$(x_2,x_2)$の加算を以下のように定義する.
\begin{equation}
  (x_1,y_1)+(x_2,y_2)=
  \left( 
    \frac{x_1y_2+x_2y_1}{1+dx_1x_2y_1y_2},
    \frac{y_1y_2+ax_1x_2}{1-dx_1x_2y_1y_2} 
  \right)
\end{equation} \\[1em]
\noindent{\large\textbf{無限遠点}}\\
\indent ツイストエドワーズ曲線で無限遠点$\mathcal{O}$は以下の性質をもつ.\\[0.5em]
\noindent\textbf{(1)加法における単位元}\\
\indent 無限遠点は楕円曲線$E$上の特別な点で任意の点$P$に対して, 次の式が成り立つ.
\begin{equation}
  P+\mathcal{O}=\mathcal{O}+P=P
\end{equation}
\indent ツイストエドワーズ曲線を使用したEd25519において, 無限遠点$\mathcal{O}$は
次の座標をもつと定義される.
\begin{equation}
  \mathcal{O}=(0,1)
\end{equation}
\indent 曲線の方程式(2.2)に$(x,y)=(0,1)$を代入すると,
\begin{equation}
  a\cdot 0^2 + 1^2 = 1 + d\cdot 0^{2}\cdot 1^2
\end{equation}
となり, 明らかに成り立つ.\\[1em]
\noindent \textbf{(2)点の逆元}\\
\indent 点$P=(x,y)$に対して, その逆元$-P$は次のように定義される.
\begin{equation}
  -P=(x,-y)
\end{equation}
\indent しかし, 無限遠点の逆元は自分自身であるため, 
\begin{equation}
  -\mathcal{O}=\mathcal{O}=(0,1)
\end{equation}
となる.\\
\noindent \textbf{(3)スカラー倍}\\
\indent スカラー倍($kP$)で$k=0$の場合, 常に無限遠点となる.
\begin{equation}
  0\cdot P=\mathcal{O}=(0,1)
\end{equation}

\indent ツイストエドワーズ曲線$E$は$\mathcal{O}$を単位元とする加法群になる.\\
楕円点の加算, 2倍算(同一の点の加算), 無限遠点の加算が同一式で
表現されるため, ECDSAのような分岐処理が不要になり, 計算が簡略化される.\\



