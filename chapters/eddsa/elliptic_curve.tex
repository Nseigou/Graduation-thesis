この節ではEd25519で使用される\textbf{ツイストエドワーズ曲線
(Twisted Edwards Curve)}について説明する.

ツイストエドワーズ曲線は体$\mathbb{F}_p$ ($p$は素数かつ$p\neq 2$)上で定義される.
非零かつ異なる要素$a,d\in \mathbb{F}_p$を持つとき、以下の式で与えられる.
\begin{equation}
  ax^2 + y^2 = 1 + dx^2y^2
\end{equation}

\paragraph{楕円加算}\leavevmode\\
点の加算式は以下の通りである.
\begin{equation}
  (x_1,y_1)+(x_2,y_2)=(\frac{x_1y_2+x_2y_1}{1+dx_1x_2y_1y_2},\frac{y_1y_2+ax_1x_2}{1-dx_1x_2y_1y_2})
\end{equation}
\vspace{1em}

$\mathcal{O}$は無限遠点と呼ばれ、
$E$は $\mathcal{O} := (0, 1)$ を単位元とする加法群になる.
エドワーズ曲線では、楕円点の加算、2倍算(同一の点の加算)、無限遠点の
加算が同一式で表現される.
このため、ECDSAのような分岐処理が不要になり、計算が簡略化される.\\



