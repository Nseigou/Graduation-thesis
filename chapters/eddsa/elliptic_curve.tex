この節ではEdDSAで使用される\textbf{ツイストエドワーズ曲線
(Twisted Edwards Curve)}について説明する.\\
\indent ツイストエドワーズ曲線は体$\mathbb{F}_p$ ($p$は素数かつ$p\neq 2$)上で定義される.
異なる非零元$a,d\in \mathbb{F}_p$ に対し, 以下の式で与えられる.
\begin{equation}
  ax^2 + y^2 = 1 + dx^2y^2
\end{equation}

\noindent\paragraph{楕円加算}\leavevmode\\
\indent 点$(x_1,y_1)$と$(x_2,x_2)$の加算式を以下のように定義する.
\begin{equation}
  (x_1,y_1)+(x_2,y_2)=
  \left( 
    \frac{x_1y_2+x_2y_1}{1+dx_1x_2y_1y_2},
    \frac{y_1y_2+ax_1x_2}{1-dx_1x_2y_1y_2} 
  \right)
\end{equation}
\vspace{1em}

\indent$\mathcal{O}$は無限遠点と呼ばれ,次の性質を持つ.\\
\noindent \textbf{加法における中立元}\\
\indent 無限遠点は楕円曲線$E$上の特別な点で任意の点$P$に対して, 次の式が成り立つ.
\begin{equation}
  P+\mathcal{O}=\mathcal{O}+P=P
\end{equation}
\indent ツイストエドワーズ曲線を使用したEd25519において, 無限遠点$\mathcal{O}$は
次の座標をもつと定義される.
\begin{equation}
  \mathcal{O}=(0,1)
\end{equation}
\indent 曲線の方程式(2.1)に$(0,1)$を代入すると,
\begin{equation}
  a(0)^2 + (1)^2 = 1 + d(0)^2(1)^2
\end{equation}
となり, 明らかに成り立つ.\\[1em]
\noindent \textbf{点の逆元}\\
\indent 点$P=(x,y)$に対して, その逆元$-P$は次のように定義される.
\begin{equation}
  -P=(x,-y)
\end{equation}
\indent 無限遠点の逆元は自分自身であるため, 
\begin{equation}
  -\mathcal{O}=\mathcal{O}
\end{equation}
となる.\\
\noindent \textbf{スカラー倍}\\
\indent スカラー倍($kP$)で$k=0$の場合, 常に無限遠点となる.
\begin{equation}
  0・P=\mathcal{O}
\end{equation}

\indent ツイストエドワーズ曲線$E$は$\mathcal{O}$を単位元とする加法群になる.\\
楕円点の加算, 2倍算(同一の点の加算), 無限遠点の加算が同一式で
表現されるため, ECDSAのような分岐処理が不要になり, 計算が簡略化される.\\



