EdDSAのアルゴリズム内では, 整数や点をオクテット列に変換するエンコードと
その逆変換であるデコードが行われる\cite{インフォーズ}.\\
 以下にED25519で使用されるデータの変換について説明する.
なお, Ed25519では$b=256$である.\\


\noindent{\large\textbf{オクテット}}\\
 8ビットからなるビット列
\[
b_0b_1b_2b_3b_4b_5b_6b_7
\]
を\emph{オクテット}と呼ぶ.
厳密には8ビット以外を指すこともある「バイト」の代わりに, 
必ず8ビットのことを指すものとして使われている語である.\\
なお, 最下位ビットは$b_0$, 最上位ビットは$b_7$である.\\[1em]
\shadowbox{例. 数値$0d128$に対応するオクテットは$00000001$である.}\\[1em]

\noindent{\large\textbf{リトルエンディアン形式}}\\
 リトルエンディアン形式とは, 数値をバイト単位で格納する際に, 
最下位バイト(数値の最小の値を持つバイト)を先頭に配置し, 
続けてその次に小さいバイトを配置していく方法を指す.
これは, 通常の十進法で右端から左へ数字を読むのと逆の順番で
データを並べることになる.\\
 例えば, 32ビットの数値 $0x12345678$ をリトルエンディアン形式で
メモリに格納すると, 次のような順番でバイトが並ぶ:
\begin{itemize}
  \item 最下位バイト(1バイト目):$0x78$
  \item 2バイト目:$0x56$
  \item 3バイト目:$0x34$
  \item 最上位バイト(4バイト目):$0x12$
\end{itemize}
\noindent このようにして, $0x12345678$ はメモリ上で 
$0x78, 0x56, 0x34, 0x12$ の順番に格納される.\\[1em]

\noindent{\large\textbf{エンコードとデコード}}\\
\begin{enumerate}
  \item ENC$(s)$\\
   整数をバイト列に変換し, データとして扱いやすくするための関数.\\
  処理:\\
   $b$ビットの整数$s$を入力として, $s$を リトルエンディアン形式にして
  $\tfrac{b}{8}$オクテットに変換して出力する.
  \item DEC$(t)$\\
   計算で使用できるよう, バイト列を元の整数に戻す関数.\\
  処理:\\
   オクテット列$t$を入力として, 整数$s$に変換して出力する.
  つまり, $\mathrm{DEC}(t)=\mathrm{ENC}^{-1}(t)=s$である.
  \item ENCE$(A)$\\
   ツイストエドワーズ曲線上の点をオクテット列に変換し, 
  $x$座標の符号も付加して効率的に表現するための関数.\\
  処理:\\
   はじめに符号関数を
  \[
    \text{sign}(x) =
    \begin{cases}
    0 & \text{if } x \geq 0 \\
    1 & \text{if } x < 0
    \end{cases}
  \]
  とする.ここで$a$は整数である.\\
   ツイストエドワーズ曲線上の点A$(x,y)$の$y$を入力として, 
  ENC$(y)$によりオクテット列$y'=(y'_1,y'_2,...,y'_\frac{b}{8})$に変換し, 
  $y'_\frac{b}{8}$の最上位ビットにsign$(x)$を格納した
  オクテット列を出力する.
  \item DECE$(t)$\\
   オクテット列を再びツイストエドワーズ曲線上の点$(x, y)$に変換する関数.
  この変換の過程で符号や整合性のチェックを行い, 正しい点を復元する.\\
  処理:\\
   オクテット列$t$を入力として, ツイストエドワーズ曲線上の点$(x,y)$に
  変換して出力する.
  \begin{enumerate}
    \item $t$の最終オクテットの最上位ビットを$x$座標の符号として取り出し
    $x_0$に格納する.($x_0=0$ または, $x_0=1$とする.)
    \item $t$の最終オクテットの最上位ビットを0に設定する.
    \item $y=$DEC$(t)$を計算し, $0\leq y<p$でないならばデコード失敗.
    \item 以下の処理を行う.
    \begin{enumerate}
      \item $u=y^2-1$, $v=d*y^2+1$として
      $x=uv^3(uv^7)^{\tfrac{p-5}{8}}$mod $p$を計算する.
      \item $vx^2 \neq \pm  u$ mod $p$ならばデコード失敗.
      \item $vx^2=-u$ mod $p$ならば, $x=2^{\tfrac{p-1}{4}}x$
    \end{enumerate}
    \item $x=0$かつ$x_0=1$ならばデコード失敗.
    \item $x_0$が$x$ mod $2$と異なるならば$x=p-x$とする.
    \item 点$(x,y)$を出力する.
  \end{enumerate}
\end{enumerate}

