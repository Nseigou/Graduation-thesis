EdDSAのアルゴリズム内では、整数や点をオクテット列に変換するエンコードと
その逆変換であるデコードが行われる.\\
以下で使用されるデータの変換について説明する.\\[0.5em]

\paragraph{オクテット}\leavevmode\\
オクテットは\textbf{$b_0b_1b_2b_3b_4b_5b_6b_7$}のような
8ビットのビット列であり、$b_0$を最下位ビット、$b_7$を最上位ビットと呼ぶ.\\[1em]
\shadowbox{例. 数値$0d128$のオクテットに対応するビット列は$00000001$である.}
\vspace{1em}
\paragraph{リトルエンディアン}\leavevmode\\
リトルエンディアン形式では、データを格納する際に
数値の下位バイト(最下位ビットに近い方)から順に配置する.\\[1em]
\shadowbox{例. 数値$0x12345678$をリトルエンディアン形式で格納すると、
$0x78, 0x56, 0x34, 0x12$となる.}
\vspace{1em}
\paragraph{エンコードとデコード}
\begin{enumerate}
  \item $ENC(s)$\\
  整数$s (0<s<L-1)$は、8ビットずつをオクテットとみなすことに基づき、
  リトルエンディアン形式で$\tfrac{b}{8}$オクテットに格納される.
  \item $DEC(t)$\\
  $t$はオクテット列であり、$ENC(s)$の逆変換によって整数$s$に変換される.
  \item $ENCE(A)$\\
  $E$の点$A$は、元$(x,y)$の$y$を$ENC(y)$によりオクテット列に変換し、
  その最終オクテットの最上位ビットに$x$座標の符号
  ($x≧0$ならば0、$x<0$ならば1)が格納される.
  \item $DECE(A)$\\
  $t$は変換元の$\tfrac{b}{8}$オクテットのオクテット列である.
  \begin{enumerate}
    \item $t$の最終オクテットの最上位ビットを$x$座標の符号として取り出し
    $x_0$に格納する.($x_0=0$ または、$x_0=1$とする.)
    \item $t$の最終オクテットの最上位ビットを0に設定する.
    \item $y=DEC(t)$を計算し、$0\leq y<p$でないならばデコード失敗.
    \item 以下の処理を行う.
    \begin{enumerate}
      \item $u=y^2-1$, $v=d*y^2+1$として
      $x=uv^3(uv^7)^{\tfrac{p-5}{8}}$mod$p$を計算する.
      \item $vx^2 \neq \pm  u$ mod$p$ならばデコード失敗.
      \item $vx^2=u$ mod$p$ならば、$x=2^{\tfrac{p-1}{4}}x$
    \end{enumerate}
  \end{enumerate}
\end{enumerate}

