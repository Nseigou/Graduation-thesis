\textbf{EdDSA}(\textbf{Edwards-curve Digital Signature Algorithm})
とは, 2011年にDaniel J. Bernsteinらによって提案された決定論的
シュノア署名\cite{schnorr}の一種であり, 効率的な点の加算公式をもつ特殊な楕円曲線である
エドワーズ曲線, またはツイストエドワーズ曲線\cite{twisted}を
利用したデジタル署名である.\cite{high-speed}
EdDSAの開発は, 従来のECDSAが持ついくつかの欠点を克服する目的で進められた.
2017年, インターネット技術やプロトコル, 標準, ソフトウェアなどに関する
公式な文書である RFC 8032 \cite{8032}が, インターネットの技術的な
標準を策定する国際的な組織 IETF によって公開され, EdDSAが標準化された.
さらに, 2019年には
% 国立標準技術研究所(National Institute of Standards and Technology, NIST)
% による情報処理標準規格 
NIST の FIPS 186-5で標準化された.
EdDSAはECDSAに比べて多くの利点があるため, 
インターネット上で通信する際にデータを安全に暗号化し, 
通信相手の認証を行うためのプロトコルである 
TLS(Transport Layer Security)1.3 \cite{rfc8446}や
ネットワーク経由で他のコンピュータと安全に通信するためのプロトコルである 
SSH, ブロックチェーン技術などの
様々なプロトコルやアプリケーションで採用されている\cite{monero}.\\
 RFC8032によると, EdDSAでは暗号攻撃に対して
約128ビットのセキュリティレベルの\textbf{Ed25519}と, 
約224ビットのセキュリティレベルの\textbf{Ed448}の
2種類の実装のどちらかを安全性要件に応じて利用するよう推奨されている.
Ed25519は2005年にBernsteinらによって提案された
Curve25519\cite{curve25519}を基盤としており, 特に
「ツイストエドワーズ曲線」という形式を採用することで, 
計算効率とセキュリティを向上させている.
その性能は2.5節で説明するように, ECDSAよりも高速であり, 
ECDSAの脆弱性も克服している.
現在, Ed25519はEdDSAの最も一般的なインスタンスであり, 
必要なリソースが少なく, 十分な安全性を提供している.
Ed448はEd25519よりもセキュリティレベルが高いが,
オーバースペックとなり, 一般的な利用には向いていない.
よって, 本研究ではEd25519を用いてデジタル署名を実装する.\\
 この章ではEd25519について概説する.
2.1節ではEd25519で使用されるデータの変換について, 
2.2節ではEd25519のパラメータについて述べる.
2.3節ではEd25519の理解に必要となる楕円曲線とその上での演算について述べる.
2.4節ではEd25519のデジタル署名アルゴリズムについて述べ,
最後に, 2.5節ではECDSAとEd25519のセキュリティと計算効率の比較を行い, 
Ed25519を用いる利点を説明する.\\
\section{ツイストエドワーズ曲線}
この節ではEd25519で使用される\textbf{ツイストエドワーズ曲線
(Twisted Edwards Curve)}について説明する.

ツイストエドワーズ曲線は体$\mathbb{F}_p$ ($p$は素数かつ$p\neq 2$)上で定義される.
非零かつ異なる要素$a,d\in \mathbb{F}_p$を持つとき、以下の式で与えられる.
\begin{equation}
  ax^2 + y^2 = 1 + dx^2y^2
\end{equation}

\paragraph{楕円加算}\leavevmode\\
点の加算式は以下の通りである.
\begin{equation}
  (x_1,y_1)+(x_2,y_2)=(\frac{x_1y_2+x_2y_1}{1+dx_1x_2y_1y_2},\frac{y_1y_2+ax_1x_2}{1-dx_1x_2y_1y_2})
\end{equation}
\vspace{1em}

$\mathcal{O}$は無限遠点と呼ばれ、
$E$は $\mathcal{O} := (0, 1)$ を単位元とする加法群になる.
エドワーズ曲線では、楕円点の加算、2倍算(同一の点の加算)、無限遠点の
加算が同一式で表現される.
このため、ECDSAのような分岐処理が不要になり、計算が簡略化される.\\




\section{データの変換}
EdDSAのアルゴリズム内では, 整数や点をオクテット列に変換するエンコードと
その逆変換であるデコードが行われる\cite{インフォーズ}.\\
 以下にEd25519で使用されるデータの変換について説明する.\\
なお, $b$は8の倍数とする. 
また, 16進数表記の数値は$0x$FFのようにその数値の前に
「$0x$」を付けることで表す.\\[1em]

\noindent{\large\textbf{ビット列}}\\
 8ビットからなるビット列
\[
b_0b_1b_2b_3b_4b_5b_6b_7
\]
を\emph{オクテット}と呼ぶ.
厳密には8ビット以外を指すこともある「バイト」の代わりに, 
必ず8ビットのことを指すものとして使われている語である.\\
ここで, 最下位ビットは$b_0$, 最上位ビットは$b_7$である.\\
\indent 例として, $0x12$は, 2進数では$00010010$であり, 
8ビットのビット列で表すと$01001000$となる.\\ 

\noindent{\large\textbf{リトルエンディアン形式}}\\
 リトルエンディアン形式とは, 数値をバイト単位で格納する際に, 
最下位バイト(数値の最小の値を持つバイト)を先頭に配置し, 
続けてその次に小さいバイトを配置していく方法を指す.
これは, 通常の十進法で右端から左へ数字を読むのと逆の順番で
データを並べることになる.\\
 例えば, 32ビットの $0x12345678$ をリトルエンディアン形式で
メモリに格納すると, 次のような順番でバイトが並ぶ:
\begin{itemize}
  \item 最下位バイト(1バイト目):$0x78$
  \item 2バイト目:$0x56$
  \item 3バイト目:$0x34$
  \item 最上位バイト(4バイト目):$0x12$
\end{itemize}
\noindent このようにして, $0x12345678$ はメモリ上で 
$0x78, 0x56, 0x34, 0x12$ の順番に格納される.\\[1em]

\noindent{\large\textbf{エンコードとデコード}}\\
\begin{enumerate}
  \item ENC$(s)$\\
   整数をオクテット列に変換し, データとして扱いやすくするための関数.\\
  処理:\\
   $b$ビットの整数$s$を入力として, $s$を リトルエンディアン形式にして
  $\tfrac{b}{8}$個のオクテットに変換して出力する.
  \item DEC$(t)$\\
   計算で使用できるよう, オクテット列を元の整数に戻す関数.\\
  処理:\\
   オクテット列$t$を入力として, 整数$s$に変換して出力する.
  つまり, $\mathrm{DEC}(t)=\mathrm{ENC}^{-1}(t)=s$である.
  \item ENCE$(A)$\\
   ツイストエドワーズ曲線上の点をオクテット列に変換し, 
  $x$座標の符号も付加して効率的に表現するための関数.\\
  処理:\\
   はじめに符号関数を
  \[
    \text{sign}(a) =
    \begin{cases}
    0 & \text{if } a \geq 0 \\
    1 & \text{if } a < 0
    \end{cases}
  \]
  とする.ここで$a$は整数である.\\
   ツイストエドワーズ曲線上の点A$(x,y)$の$y$を入力として, 
  ENC$(y)$によりオクテット列$y'=(y'_1,y'_2,...,y'_\frac{b}{8})$に変換し, 
  $y'_\frac{b}{8}$の最上位ビットにsign$(x)$を格納した
  オクテット列を出力する.
  \item DECE$(t)$\\
   オクテット列を再びツイストエドワーズ曲線上の点$(x, y)$に変換する関数.
  この変換の過程で符号や整合性のチェックを行い, 正しい点を復元する.\\
  処理:\\
   オクテット列$t$を入力として, 以下の手順でツイストエドワーズ曲線上の点$(x,y)$に
  変換して出力する.
  \begin{enumerate}
    \item[① ] $t$の最終オクテットの最上位ビットを$x$座標の符号として取り出し
    $x_0$に格納する.($x_0=0$ または, $x_0=1$とする.)
    \item[② ] $t$の最終オクテットの最上位ビットを0に設定する.
    \item[③ ] $y=$DEC$(t)$を計算し, $0\leq y<p$でないならばデコード失敗.
    \item[④ ] 以下の処理を行う.
    \begin{enumerate}
      \item $u=y^2-1$, $v=dy^2+1$として
      $x=uv^3(uv^7)^{\tfrac{p-5}{8}}\pmod p$を計算する.
      \item $vx^2 \neq \pm  u \pmod p$ならばデコード失敗とし, 処理を中断する.
      \item $vx^2=-u \pmod p$ならば, $x\leftarrow 2^{\tfrac{p-1}{4}}x$
    \end{enumerate}
    \item[⑤ ] $x=0$かつ$x_0=1$ならばデコード失敗とし, 処理を中断する.
    \item[⑥ ] $x_0$が$x \pmod 2$と異なるならば$x\leftarrow p-x$とする.
    \item[⑦ ] 点$(x,y)$を出力する.
  \end{enumerate}
\end{enumerate}


\newpage
\section{パラメータ}
EdDSAのパラメータは以下の通りである.なお, エドワーズ曲線を$E$とする.\\
\begin{table}[htbp]
  \centering
  \begin{tabular}{cp{10cm}}
    \hline
    \multicolumn{1}{c}{パラメータ} & \multicolumn{1}{c}{説明} \\ \hline \hline
    $p$ & 法となる素数. EdDSAは$\mathbb{F}_p$上の楕円曲線を使用する.\\
    $b$ & $b\geq 10$かつ$p<2^{b-1}$となる正整数. 公開鍵の長さを表す.\\
    H & ハッシュ関数. 2bビット長のハッシュ値を出力する. \\
    $a$ & $E$を決定するパラメータ. $\mathbb{F}_p$上の平方剰余. すなわち, $x^2\equiv d \pmod{p}$となる$x$が存在する.\\
    $d$ & $E$を決定するパラメータ. $a\neq d$. 非ゼロの非剰余. すなわち, $x^2\equiv d \mod{p}$となる$x$が存在しない.\\
    $c$ & $E$を決定するパラメータ. $2$または$3$. $2^{c}$はコファクタと呼ばれる.\\
    $L$ & $E$を決定するパラメータ. $2^{200}$より大きい奇素数で$E$の位数$#E=2^{c}l$となるような数であり, $B$の位数.\\
    $B$ & $E$上のベースポイント. $B\neq (0,1)$\\
    $n$ & $c\leq n < b$となる整数.\\ \hline
  \end{tabular}
  \caption{EdDSAのパラメータ}
\end{table}

本研究で使用するEd25519のパラメータは以下の通りである.\\
\begin{longtable}{cc}
  \caption{Ed25519のパラメータ}
  \endlastfoot
  \hline
  \multicolumn{1}{c}{パラメータ} & \multicolumn{1}{c}{値, または関数} \\ \hline \hline
  $p$ & $2^{255}-19$ \\
  $b$ & 256 \\
  H & SHA-512 \\
  $a$ & $-1$ \\
  $d$ & $-\frac{121665}{121666}$ \\
  $c$ & 3 \\
  $L$ & $2^{252} + 27742317777372353535851937790883648493$ \\
  $B$ & $(15112221349535400772501151409588531511454012693$ \\
  & $041857206046113283949847762202,$\\
  & $4631683569492647816942839400347516314130799386625622$ \\
  & $5615783033603165251855960)$ \\
  $n$ & 254 \\ \hline
\end{longtable}


\section{デジタル署名アルゴリズム}
Ed25519 における3つのアルゴリズムの手順を以下に述べる.\\
\indent なお, ENC, DEC, ENCE, DECEは3.2節で述べた関数を使用する.
また, $H, L, B$は3.3節で述べたEd25519の値, または関数を使用する.\\[1em]
\let\ltxlist\list
\begin{breakitembox}[l]{\textbf{鍵生成}}
   
  \begin{enumerate}[parsep=7pt]
    \item $256$バイトのランダムな値$Pk$を生成し, 秘密鍵とする.
    \item $h=H(Pk)=(h[0]\mid\mid h[1]\mid\mid \cdots\mid\mid h[63])$
    を計算する. (ここで$h[n]$はハッシュ関数の出力512ビットを
    オクテット配列と見なしたときの$n$オクテット目のビット列を表す.)
    \item $h$を前半部分 $h[0]$ から $h[31]$ と後半部分 $h[32]$ から $h[63]$ に分ける.
    \item 前半部分の最初のバイト($h[0]$)の下位3ビットを0にクリアする.
    最後のバイト($h[31]$)の最上位ビットを0に, 最上位2ビット目を1に設定したものを
    リトルエンディアン形式の整数として解釈し, スカラー $s \pmod L$を生成する. 
    ただし, スカラー値$s$は 8 の倍数で,ちょうど 255 ビットとなる.
    \item 基準点$B$を使って$A = sB$を計算し, ENCE$(A)$を公開鍵とする. 
  \end{enumerate}
\end{breakitembox}
\vspace{2em}
\let\ltxlist\list
\begin{breakitembox}[l]{\textbf{署名生成}}
   
  \begin{enumerate}[parsep=7pt]
    \item 秘密鍵$Pk$を使って, ハッシュ値 $h=H(Pk)$ を計算する.
    \item $h$の後半部分 $h[32]$ から $h[63]$ を使って, 
    \begin{center}
      $r \equiv H(h[32]\mid\mid \cdots \mid\mid h[63] \mid\mid M) \pmod L$
    \end{center}
    を計算し, r\leftarrow DEC$(r)$とする.
    \item $R=rB$ を計算し, ENCE$(R)$によりエンコードする.
    \item $k=H(R \mid\mid A \mid\mid M)$ を計算する.
    \item 鍵生成の際に計算した $s$ を用いて $S\equiv r+ks \pmod L$ を計算し, ENCE$(S)$によりエンコードする.
    \item $(R,S)$を署名とする.
  \end{enumerate}
\end{breakitembox}
\vspace{2em}
\let\ltxlist\list
\begin{breakitembox}[l]{\textbf{署名検証}}
 
  \begin{enumerate}[parsep=7pt]
    \item メッセージ$M$と署名$(R,S)$を受け取る.
    \item $R'=$DECE$(R)$ として, $R$をデコードする.
    \item $S'=$DEC$(S)$ として, $S$をデコードする.
    \item $A'=$DECE$(A)$ として, $A$をデコードする.
    \item 2~4でデコードに失敗した場合や$0\leq S<L$の範囲外である場合, 
    署名を棄却する.
    \item $k'=H(R' \mid\mid A' \mid\mid M)$ を計算する.
    \item $8S'B=8R'+8k'A'$ が成り立てば, 署名は有効である.
    そうでなければ, 署名は無効である.
  \end{enumerate}
\end{breakitembox}
% \begin{breakitembox}[l]{\textbf{署名検証}}
%  
%   \begin{enumerate}[parsep=7pt]
%     \item メッセージ$M$と署名$(R,S)$を受け取る.
%     \item 公開鍵$A$を用いて, $k'=H(R \mid\mid A \mid\mid M)$ を計算する.
%     \item $8SB=8R+8k'A$ が成り立てば, 署名は有効である.
%     そうでなければ, 署名は無効である.
%   \end{enumerate}
% \end{breakitembox}

\section{ECDSAとEd25519の比較}
Ed25519はECDSAと比べ, 以下の点でセキュリティと処理時間が向上している.\\[1em]
{\large\textbf{セキュリティ}}\\[1em]
\noindent\textbf{決定論的な署名生成}\\
 Ed25519 は, 署名ごとにランダムな値を生成する代わりに, 
メッセージと秘密鍵をハッシュすることで 
ノンス(署名に使用するランダムな数値)を決定論的に生成する.
この方法により, 乱数生成の失敗による秘密鍵の漏洩リスクを回避している.
一方, ECDSA では, 乱数が予測可能または重複した場合, 秘密鍵が簡単に
推測されるリスクがあり, これがセキュリティの大きな弱点となる.\\[1em]
\noindent{\large\textbf{処理時間}}\\[1em]
 Ed25519 は, ねじれたエドワーズ曲線を使用しており, 
この形式の曲線は楕円曲線演算を効率的に実行できる特性を持っている.
特に, 加算と倍加の操作が簡素化され, 計算ステップが少なく済むため, 
ソフトウェアでの実装が高速になる.\\

\noindent\textbf{乱数生成の回避} \leavevmode\\
 EdDSA では乱数を鍵生成でのみ使用している.この乱数は秘密鍵に直接影響を
与えるため,暗号的に安全である必要がある.一方で,署名生成,署名検証では
乱数を使用しない.一般的に,暗号的に安全な乱数は生成の時間がランダムで
処理コストが高い.EdDSA では署名生成,署名検証を一定時間で行い,
乱数を使用するアルゴリズムに比べて高速に処理することができる.\\

\noindent\textbf{乗法逆元の不要} \leavevmode\\
 Ed25519では拡張した射影座標系(拡張ツイストエドワーズ座標)で
計算することがRFC8032で推奨されている.
この座標系の特性により, 処理コストの高い乗法逆元の処理が不要となり
処理が単純化され, 高速化しやすい.

