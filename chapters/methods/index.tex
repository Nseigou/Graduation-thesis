\indent この章では, 階戸のプロトコルを更に効率化させるための改良方法と, 
それを評価する方法について述べる. \\

\noindent {\Large\textbf{改良方法とその評価方法}}\\[0.5em]
\indent 階戸は, デジタル署名方式としてDSAとECDSAを使用してシミュレーションを行い, 
その結果をもとに, 署名方式の違いによる通信への影響を評価していた. 
一方, 3章では, EdDSAがECDSAよりもセキュリティと処理時間において優れた署名方式であり, 通信性能
を向上させる可能性を示した. そこで, 本研究では, 署名方式をEdDSAに差し替えることで
階戸のプロトコルの更なる効率化を図った. さらに, EdDSAを組み込んだプロトコルの作成と, 
そのプロトコルを用いたシミュレーションを実施し, EdDSAが階戸のプロトコルにおいて, 
どのような影響を与えるかを他の署名方式と比較することで評価を行った. 
EdDSAの導入方法と評価基準の選定について, 以下に述べる.\\[0.5em]
\noindent{\large\textbf{(1) EdDSAの導入}}\\
\indent 階戸は, ns-3のGPSRのモジュールにOpen SSLのDSAとECDSAの署名機能を追加する
コーディング行っていた. 本研究では, EdDSAを導入するために, Open SSLのEd25519
\cite{openssl-eddsa}の署名機能をGPSRのモジュールに追加するようコーディングを行った. \\[0.5em]
\noindent{\large\textbf{(2) 評価基準の選定}}\\
\indent 以下の8項目の評価基準により, ECDSAとEdDSAの2つの署名方式の違いによる
性能差を評価する. なお, 遅延時間は中央値と最頻値, その他は平均値を用いる. 
\vspace{-3mm}
\setlength{\columnsep}{10pt} % デフォルトは約35pt
\begin{multicols}{2}
  \begin{enumerate}
      \item スループット
      \item 遅延時間 (Delay) 
      \item パケット配送率 (PDR)
      \item オーバーヘッドサイズ\\
      \item シミュレーション実行時間
      \item 署名作成時間
      \item 署名検証時間
      \item メモリ使用量
  \end{enumerate}
\end{multicols}