この節では, 第3章で説明したEdDSAの性能と第5章で示した実験結果から, 
EdDSAの実装について評価する. 最後に, 本研究で行った実験結果から
想定されるEdDSAの利用シーンについての考察を与える.\\
\indent 実装評価に先立ち, 実験の結果を以下にまとめておく. 
\begin{enumerate}
  \item 実験1\\
  \indent EdDSAを用いたことで, 不正ノードを排除することができたが, 
  ECDSAとの差は見られず, セキュリティ性能は同等であった. 
  \item 実験2\\
  \indent 遅延時間とオーバーヘッドサイズにおいて, EdDSAは
  ECDSAとの差は見られず, ネットワークへの負荷は同程度であった.
  \item 実験3\\
  \indent EdDSAはECDSAよりも署名に関する計算効率が良く, 
  ネットワークの拡大に対するスケーラビリティが高いことがわかった.
\end{enumerate}

\indent 上述の通り, EdDSAはECDSAよりも
計算効率が高いことがわかった. これは, 第3章で述べたEdDSAの設計上の
特長である高速な処理能力が, 実験環境においても十分に発揮されたことを
示している. また, 第3章で述べたセキュリティの堅牢性から, 
EdDSAは秘密鍵を特定しようとする攻撃者が存在する環境で, ECDSAよりも
高いセキュリティを提供できる. よって, V2Vアドホックネットワーク
において, EdDSAはECDSAよりも安全かつ効率的な認証方式として
利用できるといえる. \\[0.5em]
\indent シミュレーション実験から, EdDSAはECDSAよりもスケーラビリティに優れていることがわかった. 
しかし, 本研究の実験環境では, 署名の生成と検証の時間が
パケットの送信間隔(1秒)に比べて非常に短い. そのため,
署名に関する処理が通信時間やスループットなどの通信品質に及ぼす影響はごくわずかであり,  
ECDSAとEdDSAの間で特徴的な差がでなかったと予想される. \\
\indent 次のような環境であれば署名の生成と検証の回数がより多くなり, 
ECDSAよりもEdDSAを用いるメリットがより顕著に表れるのではないかと考えられる. 
\begin{itemize}
  \item ノード数が非常に多い
  \item パケットの送信間隔が非常に短い
  \item 使えるリソースが限られている
\end{itemize}
さらに, 次のような環境であればEdDSAのセキュリティの堅牢さを
発揮すると考えられる. 
\begin{itemize}
  \item 秘密鍵を特定しようとする攻撃者が存在する
\end{itemize} 

\indent 上記のようなシチュエーションで実験を進めていくことで,
EdDSAがVANETに及ぼす影響をより具体的に明らかにできるのではないかと予想される.  






