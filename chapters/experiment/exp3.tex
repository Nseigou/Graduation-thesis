\indent 実験3では, 認証機構の処理能力(計算効率)を評価するための実験を行った. 
不正ノードの存在しない環境でノード数を37, 74, 112, 148, 185に設定して
250回ずつシミュレーションを行い, それぞれの実行にかかった時間を調べた.\\
\indent 実験の結果は以下の通りである. \\

{\LARGE\textbf{図5.6}}\\
{\LARGE\textbf{表5.1}}\\
{\LARGE\textbf{表5.2}}\\

\indent 図5.6にはシミュレーションパターンごとのノード数による
実行時間の変化を近似してグラフ化したものを, 表5.1にはその具体的な
値を示した. さらに, 表5.2にはECDSAとEdDSAにおいて, 1回の署名作成と
1回の署名検証にかかった時間を示した. \\
\indent 図5.6, 表5.1より, EdDSA, ECDSA, DSAの順番で実行時間が
短いのがわかる. また, 表5.2に示したように, EdDSAがECDSAに比べて
署名生成, 署名検証にかかる時間が短いことから, EdDSAは
3つの署名方式の中で最も計算効率が良いということが確かめられた. 
さらに, ECDSAとEdDSAの計算効率の差はノード数が増加するほど, 
実行時間全体に与える影響が大きなっており, EdDSAが
よりスケーラビリティに優れていることを示唆している. 