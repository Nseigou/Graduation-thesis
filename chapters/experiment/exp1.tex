実験1では, 認証機構が正しく機能していることを確認するための実験を
行った. 3章で述べた2種類の不正ノードを3個ずつ用意し, 全ノードの
約8\%(5個のノード)を不正ノードに設定した. そのような環境で
250回シミュレーションを行い, スループットと平均パケット配送率(PDR)を
調べた. \\
\indent 実験の結果は以下の通りである. \\

{\LARGE\textbf{図5.1}}\\
{\LARGE\textbf{図5.2}}\\

\indent 図5.2は, 実験1におけるシミュレーションパターンごとの
平均パケット配送率を示している. 認証機構なしの場合に
\textcolor{red}{xx}\%であるのに対し, DSAでは\textcolor{red}{xx}\%, 
ECDSAでは\textcolor{red}{xx}\%, EdDSAでは\textcolor{red}{xx}\%と, 
認証機構を追加したことで平均パケット配送率が
約\textcolor{red}{xx}\%向上した. \\
\indent 図5.1は, 実験1におけるシミュレーションパターンごとの
平均スループットを示している. 認証機構なしの場合に
\textcolor{red}{xx}kbpsであるのに対し, 
DSAでは\textcolor{red}{xx}kbps, ECDSAでは\textcolor{red}{xx}kbps,
EdDSAでは\textcolor{red}{xx}kbpsと, 認証機構を追加することで
平均パケット配送率同様, スループットも向上した. \\
\indent これらの結果は, 認証機構の追加により 
不正ノードが排除されたことで, 経路選択を行う際に正当なノードのみを
選択しており,  データの窃取(転送中止)を回避できたことを示している. 
しかし, EdDSAの結果をDSAとECDSAの結果と比較するとほとんど差がないため, 
3つの署名方式が不正ノードを排除することにおいて同等の性能をもつと考えられる. 



