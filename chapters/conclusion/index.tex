本論文は7章で構成される. 第1章では, VANETやデジタル署名について
解説した. 第2章では, 階戸\cite{shinato}のプロトコルについて説明した. 
第3章では, 本研究の対象であるEdDSA(Ed25519)の特徴や
アルゴリズムについて解説した. 第4章では, 階戸のプロトコルを更に
効率化させるための改良方法と, それを評価する方法について述べた. 
第5章では, シミュレーション環境について述べ, 第6章で
シミュレーションによる実験を行い, その結果と評価を示した. 
第7章では, 第3章と第6章の内容を踏まえ, EdDSAの実装評価について議論した. \\
\indent 本研究では, EdDSAの高い計算効率および堅牢なセキュリティ性能が, 
認証機構を追加したGPSRによるV2V通信にどのような影響を与えるかを
検証した. その結果, EdDSAを使用した場合, 署名生成および
検証にかかる処理時間が他の署名方式(ECDSA, DSA)よりも短縮され, 
ネットワーク全体の計算負荷が軽減されることを明らかにした. 
また, EdDSAの性能を最大限活かすための利用シーンの考察すると, 
EdDSAはV2VアドホックネットワークやIoTが発展に伴い, 
今後ますます重要な役割を果たすと考えられる.\\
\indent 今後の課題としては, 送受信ノードの数やノードの動き, 
通信のリンク状況などの要件を変更してシミュレーションを行い, 
多様な実験環境での影響を調べ, EdDSAのVANETへの適用可能性の調査を進めるとともに, 
より効率的な認証方法の探求も進めていきたい.

% また, 階戸のプロトコルでは, 位置情報とIPアドレスの認証が独立しており, 各ノードが
% 両方の検証を行っていたが, それらを統合し, ノードの認証をより効率的に
% 行う方法の探究もしていきたい. 