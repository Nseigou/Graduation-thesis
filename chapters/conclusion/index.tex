本論文は6章で構成される. 第1章では, VANETやデジタル署名について
解説した. 第2章では, 階戸\cite{shinato}のプロトコルについて説明した. 
第3章では, 本研究の対象であるEdDSA(Ed25519)の特徴や
アルゴリズムについて解説し, EdDSAを階戸のプロトコルに導入する方法について
述べた. 第4章では, シミュレーション環境について述べ, 第5章で
シミュレーションによる実験を行い, その結果と評価を示した. 
第6章では, 第3章と第5章の内容を踏まえ, EdDSAの実装評価について議論した. \\
\indent 本研究では, EdDSAの高い計算効率および堅牢なセキュリティ性能が, 
認証機構を追加したGPSRによるV2V通信にどのような影響を与えるかを
検証した. その結果, EdDSAを使用すれば, 署名生成および
検証にかかる処理時間の短縮とリソースの低減が実現し, 計算負荷が軽減されることを明らかにした. 
また, EdDSAの性能を最大限活かすための利用シーンの考察より, 
EdDSAはV2VアドホックネットワークやIoTの発展に伴い, 
今後ますます重要な役割を果たすと考えられる.\\
\indent 今後は, ノード数、送受信ノードのペア数, ノードの動き, 不正ノードの種類, 
通信リンク状況などの要件を変更してシミュレーションを行い, 
多様な実験環境での影響を調べ, EdDSAのVANETへの適用可能性の調査を進めていきたい. 
また, 位置情報とノード情報の検証を一元化するプロトコルの検討もしていきたい. 
