本論文は6章で構成される. 第1章では, VANETやデジタル署名について
説明した. 第2章では, 本研究の対象であるEdDSA(Ed25519)の特徴や
アルゴリズムについて解説した. 第3章では, セキュアに
ルーティングを行うためにの認証機構の導入方法について解説した. 
の. 第4章では, シミュレーション環境について述べ, 第5章で
シミュレーションによる実験を行い, その結果とその評価を示した. 
第6章では, 第2章と第5章の内容を踏まえ, EdDSAの実装評価について議論した. \\
\indent 本研究では, EdDSAの高い計算効率および堅牢なセキュリティ性能が, 
認証機構を追加したGPSRによるV2V通信にどのような影響を与えるかを
検証した. その結果, EdDSAを使用した場合, 署名生成および
検証にかかる処理時間が他の署名方式(ECDSA, DSA)よりも短縮され, 
ネットワーク全体の計算負荷が軽減されることを明らかにした. 
また, EdDSAの性能を最大限活かすための利用シーンの考察から, 
EdDSAはV2VアドホックネットワークやIoTが発展していく中で, 
今後ますます重要な役割を果たすだろう.\\