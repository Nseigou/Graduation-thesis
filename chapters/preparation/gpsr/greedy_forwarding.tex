Greedy ForwardingはGPSRの基本的なルーティング方式である.
図1.3のように, 送信ノードSは, 宛先ノードDの位置情報をもとに
自身の電波伝搬範囲内のノードから, Dに最も近いノードを
ネクストホップとして選択する. ここで, 送信ノードSは宛先ノードDの
位置情報を事前に把握していることを前提とする. 点線で書かれた円は全ノードの
受信感度が等しい場合の送信ノードSの電波伝搬範囲を, 
破線は宛先ノードとの距離を表している.\\

{\LARGE Greedyの図1.3を挿入}

Greedy Forwardingには\textbf{局所最大問題}が存在している. 
局所最大問題とは, 図1.4のように, 送信ノードSの電波伝搬範囲内に
宛先ノードDが存在しない, かつ, 送信ノードSが自身の
電波伝搬範囲内で宛先ノードDに最も近い場合, 選択できる
ネクストホップが存在しなくなるという問題である.\\

{\LARGE 局所最大問題の図1.4を挿入}
