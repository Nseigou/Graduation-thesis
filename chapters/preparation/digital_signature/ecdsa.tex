\indent \textbf{ECDSA (Elliptic Digital Signature Algorithm)}は, 
DSAを楕円曲線暗号(Elliptic Curve Cryptography, ECC)を基盤として
改良したデジタル署名アルゴリズムである.
ECDSAは, 2000年に FIPS 186-2として標準化され, 
最新版であるFIPS 186-5\cite{fips186-5}でも採用されている.
ECDSAは, DSAよりも短い鍵長で同等の安全性を確保できるため, 
計算量が少なく, 低性能なデバイスでも高速な処理が可能であり, 
かつ演算規則の複雑さから攻撃者が鍵を推測することも困難である. 
よって, 処理速度と安全性において有限体上のDSAよりも優れている. \\
\indent この項では, ECDSAの理解に必要となる楕円曲線とその上での演算について
説明し, その後, ECDSAのアルゴリズムについて述べる.\\[1em]

\noindent{\large\textbf{楕円曲線}}\\
\indent 体$\mathbb{F}_p$($p$は素数)上で定義された楕円曲線とは以下の式で与えられる
代数曲線である.
\[
  y^2+a_1y+a_3y=x^3+a_2x^2+a_4x+a_6  (a_1,a_2,a_3,a_4,a_6\in\mathbb{F}_p)
\]
特に, $p\neq 2,3$の場合, 以下の標準形に簡略化できることが知られている. 
\begin{equation}
  y^2=x^3+ax+b  (a,b\in\mathbb{F}_p)
\end{equation}
この式をWeierstrass方程式という. \\
\indent 楕円曲線(式1.1)の判別式$\Delta$が
\begin{equation}
  \Delta=-16(4a^3+27b^2)\neq 0
\end{equation}
を満たすとき, 楕円曲線(式1.1)は\textbf{非特異}であるという. 
非特異である楕円曲線は, 尖点や自己交差点, 孤立点を持たないため, 
楕円曲線が非特異であることは, 後述する楕円曲線上の点に対する演算
(加算やスカラー倍)が矛盾なく定義されるために必要不可欠である.\\
\indent 楕円曲線上の点$P=(x,y)$のうち$x,y$がともに有限体
$\mathbb{F}_p$の元であるものを\textbf{有理点}という. また, 楕円曲線上には
\textbf{無限遠点}と呼ばれる特殊な点$\mathcal{O}$の存在を仮定し, これも楕円曲線上の
有理点の集合に含める. 有理点の集合は, 以下で述べる演算に関して
群の構造をもつ. 楕円曲線暗号方式では,この性質が利用される.\\[1em]
\noindent\textbf{楕円曲線上の点の加算}\\
\indent 楕円曲線上の点$P=(x_1,y_1)$と$Q=(x_2,y_2)$の加算 $P+Q$ を
以下のように定義する.
\begin{enumerate}
  \item[(i) ] 2点$P,Q$を通る直線が$y$軸と平行でない場合(図1.6):
  \begin{enumerate}
    \item[1. ] 点$P,Q$を通る直線$L$を引く.
    \[
      L : y-y_1 = \frac{y_2-y_1}{x_2-x_1}(x-x_1)
    \]
    \item[2. ] $L$と楕円曲線$E$の交点を$R(x_3,y_3)$とする.
    \[
    \begin{aligned}
      x_3 &= \left(\frac{y_2-y_1}{x_2-x_1}\right)^2-x_1-x_2\\
      y_3 &= \frac{y_2-y_1}{x_2-x_1}(x_1-x_3)-y_1
    \end{aligned}
    \]
    \item[3. ] $R$の$x$軸対称な点$R'(x_3,-y_3)$が$P+Q$となる.
    \[
      P+Q=R'
    \]
  \end{enumerate}
  {\LARGE\textbf{図1.6を挿入}}\\
  \item[(ii) ] 2点$P,Q$を通る直線$L$が$y$軸と平行である場合:\\
  \indent この場合, 直線$L$は2点$P,Q$以外で楕円曲線$E$と交わらない.\\ 
  \indent 無限遠点$\mathcal{O}$が$P+Q$となる.
  \[
    P+Q=\mathcal{O}
  \]
  \item[(iii) ] 2点$P,Q$が同一の点である場合(図1.7):\\
  \begin{enumerate}
    \item[1. ] 点$P$における接線$L'$を引く. 
    \[
      L' : y-y_1 = \frac{3x_1^2 + a}{2y_1}(x - x_1)
    \] 
    \item[2. ] $L'$と楕円曲線$E$の交点を$R(x_3,y_3)$とする.
    \[
    \begin{aligned}
      x_3 &= \left(\frac{3x_1^2 + a}{2y_1}\right)^2-2x_1\\
      y_3 &= \frac{3x_1^2 + a}{2y_1}(x_1-x_3)-y_1
    \end{aligned}
    \]
    \item[3. ] $R$の$x$軸対称な点$R'(x_3,-y_3)$が$P+Q$となる.
    \[
      P+Q=2P=R'
    \]
  \end{enumerate}
  {\Huge\textbf{図1.7を挿入}}
  \item[(iv) ] $P=\mathcal{O}$または$Q=\mathcal{O}$である場合:\\
  \indent 無限遠点は加算において加法単位元の役割を果たす.
  \[
  \begin{aligned}
    P+\mathcal{O}&=P\\
    \mathcal{O}+Q&=Q
  \end{aligned}
  \]    
\end{enumerate}
\vspace{0.5em}
\noindent\textbf{スカラー倍算}\\
\indent 正整数$k$による点$G$のスカラー倍$P=kG$は, 点$G$を$k$回加算した結果を表す.
\[
  P=kG=\underbrace{G+G+\cdots+G}_{k\text{回}}
\]
\indent ある基準点$G$に対し, $P = kG$となる楕円曲線上の点$P$が与えられたとき, 
$k$と$G$から$P$を求めるのは容易である. しかし, 逆に$G$と$P$から$k$を
求めるのは計算量的に困難であることが知られている. 
これを\textbf{楕円曲線上の離散対数問題(ECDLP: Elliptic Curve Discrete Logarithm Problem)}という.\\[1em]

\noindent{\large\textbf{ECDSAのアルゴリズム}}\\
\indent ECDSAの3つのフェーズにおける処理は以下の通りである.
\vspace{1em}
\let\ltxlist\list
\begin{breakitembox}[l]{\textbf{鍵生成}}
   
  \begin{enumerate}[parsep=7pt]
    \item 法となる素数$p$と, 楕円曲線$E$を選び, $E$上の基準点$G$を選ぶ.
    \item $d$を$2\leq d\leq n-1$の範囲からランダムに選び, 
    署名鍵(秘密鍵)として安全に管理する. ただし, $n$は$G$の位数である.
    \item $Q=dG$を計算し, $Q$を検証鍵(公開鍵)とする.
  \end{enumerate}
\end{breakitembox}
\vspace{1em}
\let\ltxlist\list
\begin{breakitembox}[l]{\textbf{署名生成}}
   
  \begin{enumerate}[parsep=7pt]
    \item $k$を$2\leq d\leq n-1$の範囲からランダムに選び, $kG$の
    $x$座標を$r$とする.
    \item メッセージ$M$に対し, ハッシュ関数$H$を用いて, $h=H(M)$を計算する.
    \item $s=k^{-1}(h+dr)\pmod n$を計算する.
    \item $(r,s)$をメッセージ$M$に対する署名とし, 
    $M$とともに受信者に送信する.
  \end{enumerate}
\end{breakitembox}
\vspace{1em}
\let\ltxlist\list
\begin{breakitembox}[l]{\textbf{署名検証}}
   
  \begin{enumerate}[parsep=7pt]
    \item $(M,(r,s))$を受け取り, $h=H(M)$を計算する. 
    \item $u\equiv s^{-1}h\pmod n$を計算する.
    \item $v\equiv s^{-1}r\pmod n$を計算する.
    \item 楕円曲線上の点として,
    \begin{equation}
      Q'=(x',y')=uG+vQ
    \end{equation}
    を計算する.
    \item $Q'$の$x$座標$x'$が$r$と一致すれば, 署名を受理する.
    そうでなければ不正な署名とみなし, 棄却する.
  \end{enumerate}
\end{breakitembox}
\vspace{1em}
\indent ECDSAは次の式で$r=x'$が成立し, 正しく機能することが確かめられる.\\
\begin{align*}
  Q'  &= uG + vQ\\
      &= s^{-1}hG + s^{-1}rQ \\
      &= s^{-1}hG + s^{-1}rdG \\
      &= s^{-1}hG + s^{-1}(sk-h)G \\
      &= s^{-1}(hG + skG - hG) \\
      &= s^{-1}skG \\
      &= kG
\end{align*}
