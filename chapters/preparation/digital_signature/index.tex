デジタル署名とは, メッセージ作成者の否認防止を含めた, 作成者の認証を
可能にする暗号技術である. 
デジタル署名の目的は, いわゆる手書きの署名や捺印と同じ機能を電子的に達成
することといえる. また, デジタル署名の検証によって, デジタルデータが通信途中で改ざん
されていないことを保証できるため, データの信頼性を確保することができる.\\
% \textbf{デジタル署名}とは, 電子データの真正性と完全性を検証するための
% 暗号技術であり, 電子データの送信者を認証するとともに, 
% そのデータが通信途中で改ざんされていないことを保証する仕組みである. 
% この技術は, データの信頼性を確保し, 送信者がデータの送信を否定できない
% 非否認性を提供する.\\
\indent 一般に, デジタル署名は以下の3つのフェーズから構成される.

\begin{enumerate}
  \item \textbf{鍵生成}\\
  \indent 署名者(送信者)が署名に使用する鍵のペア, 
  すなわち検証鍵(公開鍵)と署名鍵(秘密鍵)を生成するフェーズである. 
  署名鍵は署名者が厳重に管理し, 秘密に保管する. 一方, 検証鍵は
  受信者や第三者に公開され, 署名の検証に用いられる.
  \item \textbf{署名生成}\\
  \indent メッセージ作成者(送信者)は, 署名対象のメッセージから
  ハッシュ関数を用いてハッシュ値を計算する. このハッシュ値は, 
  メッセージがわずかでも変更されると全く異なる値となるため, 
  メッセージが改ざんされていないこと(完全性)を検証するための
  重要な要素である. 続いて, メッセージ作成者は自身の署名鍵(秘密鍵)を
  使ってハッシュ値に署名を行う. 署名鍵がメッセージ作成者
  しか保持していないことと, メッセージ作成者以外が署名鍵を求めることは
  計算量的に困難なことから, 署名はメッセージの作成者が正当であること
  (真正性)を証明する役割を果たす. 
  \item \textbf{署名検証}\\
  \indent 検証者(受信者)は, 署名者が公開した検証鍵(公開鍵)を用いて
  署名を検証する. この検証プロセスにより, 署名が署名鍵と対になる
  検証鍵によって生成されたことを確認できるため, メッセージが正規の
  作成者によって署名され, 送信後に改ざんされていないことが保証される.
\end{enumerate}

デジタル署名にはさまざまなアルゴリズムが存在する. その中でも,    
\textbf{国立標準技術研究所(National Institute of Standards and Technology, NIST)}
による情報処理標準規格 FIPS に基づいて設計された
\textbf{DSA(Digital Signature Algorithm)}およびその改良版である
\textbf{ECDSA(Elliptic Curve Digital Signature Algorithm)}は, 
V2V通信のようなリアルタイム性やリソース制約が求められる環境に
適している. よって, 本研究ではこの2つを用いてデジタル署名を実装する.\\
以下に, DSAとECDSAの概要を示す. \\[1em] 


\noindent {\Large\textbf{DSA}}\\
\textbf{DSA (Digital Signature Algorithm)}は, 1993 年に FIPS 186-4として
標準化された, DSS (Digital Signature Standard) の主要なアルゴリズムの
1つであった. なお, 2023 年の FIPS 186-5\cite{fips186-5} では,DSAは
新たにデジタル署名を行うことには推奨されないが, 標準策定以前に行われた
署名の検証には引き続き利用可能とされている.\\
\indent DSAの3つのフェーズにおける処理は以下の通りである.\\[0.5em]
\let\ltxlist\list
\begin{breakitembox}[l]{\textbf{鍵生成}}
   
  \begin{enumerate}[parsep=7pt]
    \item セキュリティパラメータとして整数 $L,N (L>N)$ を定める. 
    FIPS186-4では以下の4つの値の組が規定されている.
    \begin{center}
      $(L,N) = (1024,160), (2048,224), (2048,256), (3072,256)$
    \end{center}
    \item $N$ビットのランダムな素数$q(2^{N-1}<q<2^{N})$, 
    $L$ ビットのランダムな素数$p(2^{L-1}<q<2^{L})$ を選ぶ.
    ただし, $q$は$p-1$を割り切る素数とする.
    \item ランダムな整数$a<p-1$に対し,
    \[
      g\equiv a^{\frac{p-1}{q}}\pmod p
    \] 
    を計算する. ただし,$g=1$である場合は再度$a$を選び直す.
    \item ランダムな整数$x(1<x<q)$に対し,
    \[begin{equation}]
      y\equiv g^x\pmod p
    \]
    を計算する.
    \item $p,q,g$は専用のパラメータとして, $y$は検証鍵(公開鍵)として
    公開する. $x$は署名鍵(秘密鍵)として安全に管理する.
  \end{enumerate}
\end{breakitembox}
\vspace{1em}
\indent 任意長のメッセージ $M$ に対して, パラメータ$p, q, g$, 秘密鍵$x$, 
ハッシュ関数$H$ を用いて, 次のように署名を生成する.
\vspace{1em}
\let\ltxlist\list
\begin{breakitembox}[l]{\textbf{署名生成}}
   
  \begin{enumerate}[parsep=7pt]
    \item ランダムな整数 $k(1<k<q)$ を選ぶ.
    \item $r\equiv (g^k\pmod p)\pmod q$ を計算する. 
    ただし, $r=0$ の場合は再度 $k$ を選び直す.
    \item $s\equiv k^{-1}(H(M+xr))\pmod q$ を計算する.
    ただし, $s=0$ の場合は再度 $k$ を選び直す.
    \item $(r,s)$ をメッセージ$M$に対する署名とし, 
    $M$とともに受信者に送信する.
  \end{enumerate}
\end{breakitembox}
\vspace{1em}
メッセージ$M$に対する署名$(r, s)$を検証するには, パラメータ$(p,q,g)$, 
公開鍵$y$, ハッシュ関数$H$を用いて, 以下のアルゴリズムを実行する.
\vspace{1em}
\let\ltxlist\list
\begin{breakitembox}[l]{\textbf{署名検証}}
   
  \begin{enumerate}[parsep=7pt]
    \item 受け取った$(r,s)から, $$0<r<q$ かつ $0<s<q$ であることを確認する. 
    これを満たさない場合は署名を棄却する.
    \item $w\equiv s^{-1}\pmod q$ を計算する.
    \item $u_1\equiv wH(M)\pmod q$を計算する.
    \item $u_2\equiv rw\pmod q$ を計算する.
    \item $v\equiv (g^{u_1}y^{u_2}\pmod p)\pmod q$ を計算する.
    \item $v=r\pmod  q$ であれば, 署名を受理する. そうでなければ
    不正な署名とみなし, 棄却する.
  \end{enumerate}
\end{breakitembox}
\vspace{1em}
\indent 正当なメッセージと署名の組$(M, (r, s))$に対し, 
\[
  g^{u_1}y^{u_2}\equiv g^{u_1+xu_2} \equiv g^{H(M)+xr}w\equiv g^{k}\pmod p
\]
が成り立つ. これは 
\begin{align*}
  v &\equiv (g^{u_1}y^{u_2} \mod p) \mod q \\
    &\equiv (g^k \mod p) \mod q \\
    &\equiv r \mod q
\end{align*}
であることを示す.\\
\indent ここで, メッセージや署名に対し改ざんが行われたとしよう. 
攻撃者が, 署名検証条件 $v \equiv r \pmod{q}$ を
満たすように操作できた場合, その攻撃は成功したとみなされる. ただし, 
署名生成に使用される値には, 署名者しか知らない秘密の乱数$k$が
含まれており, この$k$は離散対数問題に基づいて計算されるため, 
十分なビット長を持つ$p$および$q$の下では, 改ざんを成功させることは
計算量的に極めて困難である.さらに, 署名生成プロセスではメッセージの
ハッシュ値$H(M)$が使用されており, このハッシュ値はハッシュ関数の
衝突困難性に依存している. したがって, 攻撃者が異なるメッセージ $M'$ を
生成し, そのハッシュ値が元のメッセージ $M$ と同じ $H(M') = H(M)$ と
なるようにすることも困難である. この性質により, 改ざんされたメッセージ
$M'$のハッシュ値は$H(M')\neq H(M)$となり, 
署名検証条件 $v \equiv r \pmod{q}$ が成立しなくなる.
また, 攻撃者がメッセージ$M$をそのままにして署名$(r, s)$を
改ざんした場合でも, 署名にはハッシュ値$H(M)$が埋め込まれているため, 
署名検証条件は満たされない. この結果, 署名検証アルゴリズムの最終段階で
署名が有効と判定される場合, 署名$(r, s)$もメッセージ$M$も
改ざんされていないことが保証される.



\noindent {\Large\textbf{ECDSA}}\\
\indent \textbf{ECDSA (Elliptic Digital Signature Algorithm)}は, 
DSA\cite{fips186-5}を楕円曲線暗号(Elliptic Curve Cryptography, ECC)を基盤として
改良したデジタル署名アルゴリズムである.
ECDSAは, 2000年に FIPS 186-2として標準化され, 
最新版であるFIPS 186-5\cite{fips186-5}でも採用されている.
ECDSAは, DSAよりも短い鍵長で同等の安全性を確保できるため, 
計算量が少なく, 低性能なデバイスでも高速な処理が可能であり, 
かつ演算規則の複雑さから攻撃者が鍵を推測することも困難である. \\
\indent この項では, ECDSAの理解に必要となる楕円曲線とその上での演算について
説明し, その後, ECDSAのアルゴリズムについて述べる.\\[1em]

\noindent{\large\textbf{楕円曲線}}\\
\indent 体$\mathbb{F}_p$($p$は素数)上で定義された楕円曲線とは以下の式で与えられる
代数曲線である.
\[
  y^2+a_1y+a_3y=x^3+a_2x^2+a_4x+a_6  (a_1,a_2,a_3,a_4,a_6\in\mathbb{F}_p)
\]
この式をWeierstrass方程式という\cite{安田}. 
特に, $p\neq 2,3$の場合, 以下の標準形に簡略化できることが知られている. 
\begin{equation}\label{weierstrass}
  y^2=x^3+ax+b  (a,b\in\mathbb{F}_p)
\end{equation}
\indent 楕円曲線(式(\ref{weierstrass}))の判別式$\Delta$が
\[
  \Delta=-16(4a^3+27b^2)\neq 0
\]
を満たすとき, 楕円曲線(式\ref{weierstrass})は\textbf{非特異}であるという. 
非特異である楕円曲線は, 尖点や自己交差点, 孤立点を持たないため, 
楕円曲線が非特異であることは, 後述する楕円曲線上の点に対する演算
(加算やスカラー倍)が矛盾なく定義されるために必要不可欠である.\\
\indent 楕円曲線上の点$P=(x,y)$のうち$x,y$がともに有限体
$\mathbb{F}_p$の元であるものを\textbf{有理点}という. また, 楕円曲線上には
\textbf{無限遠点}と呼ばれる特殊な点$\mathcal{O}$の存在を仮定し, これも楕円曲線上の
有理点の集合に含める. 有理点の集合は, 以下で述べる演算に関して
群の構造をもつ. 楕円曲線暗号方式では,この性質が利用される.\\[1em]
\noindent\textbf{楕円曲線上の点の加算}\\
\indent 楕円曲線上の点$P=(x_1,y_1)$と$Q=(x_2,y_2)$の加算 $P+Q$ を
以下のように定義する.
\begin{enumerate}
  \item[(i) ] 2点$P,Q$を通る直線が$y$軸と平行でない場合%(図1.6):
  \begin{enumerate}
    \item[1. ] 点$P,Q$を通る直線$L$を引く.
    \[
      L : y-y_1 = \frac{y_2-y_1}{x_2-x_1}(x-x_1)
    \]
    \item[2. ] $L$と楕円曲線$E$の交点を$R(x_3,y_3)$とする.
    \[
    \begin{aligned}
      x_3 &= \left(\frac{y_2-y_1}{x_2-x_1}\right)^2-x_1-x_2\\
      y_3 &= \frac{y_2-y_1}{x_2-x_1}(x_1-x_3)-y_1
    \end{aligned}
    \]
    \item[3. ] $R$の$x$軸対称な点$R'(x_3,-y_3)$が$P+Q$となる.
    \[
      P+Q=R'
    \]
  \end{enumerate}
  % {\LARGE\textbf{図1.6を挿入}}\\
  \item[(ii) ] 2点$P,Q$を通る直線$L$が$y$軸と平行である場合:\\
  \indent この場合, 直線$L$は2点$P,Q$以外で楕円曲線$E$と交わらない.\\ 
  \indent 無限遠点$\mathcal{O}$が$P+Q$となる.
  \[
    P+Q=\mathcal{O}
  \]
  \item[(iii) ] 2点$P,Q$が同一の点である場合%(図1.7):\\
  \begin{enumerate}
    \item[1. ] 点$P$における接線$L'$を引く. 
    \[
      L' : y-y_1 = \frac{3x_1^2 + a}{2y_1}(x - x_1)
    \] 
    \item[2. ] $L'$と楕円曲線$E$の交点を$R(x_3,y_3)$とする.
    \[
    \begin{aligned}
      x_3 &= \left(\frac{3x_1^2 + a}{2y_1}\right)^2-2x_1\\
      y_3 &= \frac{3x_1^2 + a}{2y_1}(x_1-x_3)-y_1
    \end{aligned}
    \]
    \item[3. ] $R$の$x$軸対称な点$R'(x_3,-y_3)$が$P+Q$となる.
    \[
      P+Q=2P=R'
    \]
  \end{enumerate}
  % {\Huge\textbf{図1.7を挿入}}
  \item[(iv) ] $P=\mathcal{O}$または$Q=\mathcal{O}$である場合:\\
  \indent 無限遠点は加算において加法単位元の役割を果たす.
  \[
  \begin{aligned}
    P+\mathcal{O}&=P\\
    \mathcal{O}+Q&=Q
  \end{aligned}
  \]    
\end{enumerate}
\vspace{0.5em}
\noindent\textbf{スカラー倍算}\\
\indent 正整数$k$による点$G$のスカラー倍$P=kG$は, 点$G$を$k$回加算した結果を表す.
\[
  P=kG=\underbrace{G+G+\cdots+G}_{k\text{回}}
\]
\indent ある基準点$G$に対し, $P = kG$となる楕円曲線上の点$P$が与えられたとき, 
$k$と$G$から$P$を求めるのは容易である. しかし, 逆に$G$と$P$から$k$を
求めるのは計算量的に困難であることが知られている. 
これを\textbf{楕円曲線上の離散対数問題(Elliptic Curve Discrete Logarithm Problem, ECDLP)}という.\\[1em]

\noindent{\large\textbf{ECDSAのアルゴリズム}}\\
\indent ECDSAの3つのフェーズにおける処理は以下の通りである.
\vspace{1em}
\let\ltxlist\list
\begin{breakitembox}[l]{\textbf{鍵生成}}
   
  \begin{enumerate}[parsep=7pt]
    \item 法となる素数$p$と, 楕円曲線$E$を選び, $E$上の基準点$G$を選ぶ.
    \item $d$を$2\leq d\leq n-1$の範囲からランダムに選び, 
    署名鍵(秘密鍵)として安全に管理する. ただし, $n$は$G$の位数である.
    \item $Q=dG$を計算し, $Q$を検証鍵(公開鍵)とする.
  \end{enumerate}
\end{breakitembox}
\vspace{1em}
\let\ltxlist\list
\begin{breakitembox}[l]{\textbf{署名生成}}
   
  \begin{enumerate}[parsep=7pt]
    \item $k$を$2\leq k\leq n-1$の範囲からランダムに選び, $kG$の
    $x$座標を$r$とする.
    \item メッセージ$M$に対し, ハッシュ関数$H$を用いて, $h=H(M)$を計算する.
    \item $s\equiv k^{-1}(h+dr)\pmod n$を計算する.
    \item $(r,s)$をメッセージ$M$に対する署名とし, 
    $M$とともに受信者に送信する.
  \end{enumerate}
\end{breakitembox}
\vspace{1em}
\let\ltxlist\list
\begin{breakitembox}[l]{\textbf{署名検証}}
   
  \begin{enumerate}[parsep=7pt]
    \item $(M,(r,s))$を受け取り, $h=H(M)$を計算する. 
    \item $u\equiv s^{-1}h\pmod n$, $v\equiv s^{-1}r\pmod n$を計算する.
    \item 楕円曲線上の点として, 
    \[
      Q'=(x',y')=uG+vQ
    \]
    を計算する.
    \item $Q'$の$x$座標$x'$が$r$と一致すれば, 署名を受理する.
    そうでなければ不正な署名とみなし, 棄却する.
  \end{enumerate}
\end{breakitembox}
\vspace{1em}
\indent 正当なメッセージと署名の組$(M, (r, s))$に対し, 
\begin{align*}
  Q'  &= uG + vQ\\
      &= s^{-1}hG + s^{-1}rQ \\
      &= s^{-1}hG + s^{-1}rdG \\
      &= s^{-1}hG + s^{-1}(sk-h)G \\
      &= s^{-1}(hG + skG - hG) \\
      &= s^{-1}skG \\
      &= kG
\end{align*}
が成り立つため, $Q'$の$x$座標が$r$と一致するかどうかを確認することで
署名検証が可能である.\\