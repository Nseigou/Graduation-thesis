 実験に導入したEd25519のプロトコル内で使用されるリトルエンディアン、
エンコーディング、プルーニングについて説明する.\\
\begin{mdframed}[linecolor=black,roundcorner=10pt]
  \textbf{リトルエンディアン}
  \begin{enumerate}
    \item[(1)]  最下位バイトから順に配置する形式.
    プロトコル内では、秘密スカラーの生成や公開鍵の生成において、
    リトルエンディアンの整数を使用する.
  \end{enumerate}
  \textbf{エンコーディング}
  \begin{enumerate}
    \item[(1)] すべての値はオクテット文字列としてコード化され、
    整数はリトルエンディアン規則を使用してコード化される.
    \item[(2)] 楕円曲線上の点のエンコード\\
    $y$座標をリトルエンディアン形式の32オクテット文字列にエンコードし、
    32バイト目の最上位ビットを0に設定する.
    $x$座標の最下位ビットをy座標の32バイト目の最上位ビットに埋め込む.
  \end{enumerate}
  \textbf{プルーニング(ビット操作)}
  \begin{enumerate}
    \item [(1)] 最初のバイトの下位3ビットを0にクリアする.
    \item [(2)] 最後のバイトの最上位ビットを0に設定し、
                 最上位2ビット目を1に設定する.
  \end{enumerate}
\end{mdframed}
\vspace{2em}

 EdDSAの3つのアルゴリズムの手順を以下に述べる.

\begin{itembox}[l]{鍵生成}
  \begin{enumerate}[parsep=7pt]
    \item 法とする素数$p$、楕円曲線$E$、基準点$G$、
                  鍵のサイズ$b$、ハッシュ関数$H$、コファクター$c$、
                  位数$L$を定める.
    \item $b$バイトのランダムな値$sk$を生成し、秘密鍵とする.
    \item $h = H(sk)$を計算し、$h$(オクテット文字列)を前半部分h[0]からh[31]と後半部分h[32]からh[63]に分ける.
    \item 前半部分$s[0]$から$s[31]$を使ってプルーニングしたものをリトルエンディアンの整数として解釈し、スカラー$s$ (mod $L$)を生成する.
    \item 基準点$G$を使って$A = sG$を計算し、Aのエンコードを公開鍵とする.
  \end{enumerate}
\end{itembox}
  
\begin{itembox}[l]{署名生成フェーズ}
  \begin{enumerate}[parsep=7pt]
    \item 秘密鍵$sk$を使って、ハッシュ値$h=H(sk)$を計算する.
    \item $h$の後半部分$h[32]$から$h[63]$を使って、$r = DEC(H())$.
    \item
    \item
  \end{enumerate}
\end{itembox}

\begin{itembox}[l]{署名検証フェーズ}
  \begin{enumerate}[parsep=7pt]
    \item
    \item
  \end{enumerate}
\end{itembox}