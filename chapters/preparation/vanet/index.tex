\textbf{VANET (Vehicle Ad Hoc Network)}とは, モバイル
アドホックネットワーク技術を車両間通信に応用したネットワークである
\cite{adhoc,vanet}. VANETは, 車両間の通信(Vehicle-to-Vehicle, 
\textbf{V2V}), および車両インフラ(路側機やドローン)間通信
(Vehicle-to-Infrastructure, \textbf{V2I})\cite{drone}で構成され, 
固定されたインフラに依存せず, ノード同士が自律的に
通信ネットワークを形成する. VANETでは, 車両間の距離が
無線通信の範囲を超えることが一般的であるため, 図\ref{fig:vanet}のように, 
データを送信元から宛先まで直接通信できない場合に, 
中継ノードを経由してデータを転送する. このような通信を
\textbf{マルチホップ通信}という.

\begin{figure}
  \centering
  \includegraphics[scale=0.6]{figures/vanet.png}
  \caption{マルチホップ通信}
  \label{fig:vanet}
\end{figure}

VANETには次の5つの特徴がある.\\[0.5em]
\noindent\textbf{(1) 十分な電力供給}\\
\indent VANETを利用する際, 車両は走行していることを前提とするため, 
スマートフォンのような電池駆動のモバイルデバイスに比べて電力の制約を
受けないと仮定する. したがって, 長時間の稼働や高い通信レートの実現が
可能であると仮定する. しかし, 高性能な通信モジュールやセンサーを
多数搭載する場合, 車両のエネルギー効率に影響を与える可能性が考えられるため, 
効率的なデバイス設計が求められる.\\[1em]
\noindent\textbf{(2) 位置情報の取得}\\
\indent 車両はGPSを搭載しているため, 自身の位置情報を取得できる.\\[1em]
\noindent\textbf{(3) ネットワークトポロジーの急速な変化}\\
\indent 無線通信では,ノード同士が直接的に通信可能であることを
\textbf{接続している}といい, この接続状態を基にネットワーク全体の構造が形成される. 
このネットワーク構造を\textbf{ネットワークトポロジー}という. 
VANETでは通信するノードを車両と想定しているため, 
移動速度の速いノードが動的にネットワークを形成する. 
そのため, 接続の頻繁な確立と切断が発生し, 
ネットワークトポロジーは急速に変化する. \\[1em]
\noindent\textbf{(4) 移動の制約}\\
\indent 車両の動きは道路や建造物などの物理的構造に従う.\\[1em]
\noindent\textbf{(5) 安全に関する情報のリアルタイム性}\\
\indent 交通事故や道路状況に関する情報を即座に共有するためには, 
低遅延かつ信頼性の高い通信が求められる.\\

VANETは, 車両間通信や車両インフラ間通信を実現するための
優れた技術としてのポテンシャルを有する一方で, 
いくつかの課題も抱えている.
その中でも, セキュリティに関する課題は, VANETを安全かつ信頼性の
高いシステムとして運用する上で最も重要な課題の1つとなっている
\cite{vanet-challenge,vanet-security}.\\

