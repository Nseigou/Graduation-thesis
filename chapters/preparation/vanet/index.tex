\textbf{VANET(Vehicle Ad Hoc Network)}とは, モバイル
アドホックネットワーク技術を車両間通信に応用したネットワークである
\cite{adhoc, vanet}. VANETは, 車両間の通信(Vehicle-to-Vehicle, 
V2V), および車両とインフラ(路側機やドローン)間通信
(Vehicle-to-Infrastructure, V2I)\cite{drone}で構成され, 
固定されたインフラストラクチャに依存せず, ノード同士が自律的に
通信ネットワークを形成する. VANETでは, 車両間の距離が
無線通信の範囲を超えることが一般的であるため, 図1.1のように, 
データを送信元から宛先まで直接通信できない場合に, 
中継ノードを経由してデータを転送する. このような通信を
\textbf{マルチホップ通信}という.\\


{\Huge 図1.1を挿入}\\

VANETには次の5つの特徴がある.\\[0.5em]
\noindent\textbf{(1) 十分な電力供給}\\
\indent VANETを利用する際, 車両は走行していることを前提とするため, 
継続的に電力が供給される. そのため, スマートフォンのような
電池駆動のモバイルデバイスに比べて電力の制約をほとんど受けない. 
これより, 長時間の稼働や高い通信レートの実現が可能である. 
しかし, 高性能な通信モジュールやセンサーを多数搭載する場合, 
車両のエネルギー効率に影響を与える可能性があるとして, 
効率的なデバイス設計が必要である.\\[1em]
\noindent\textbf{(2) 自身の位置情報}\\
\indent 車両はGPSを搭載しているため,  自身の位置情報を取得できる.\\[1em]
\noindent\textbf{(3) ネットワークトポロジーの急速な変化}\\
\indent 無線通信では,ノード同士が直接的に通信可能であることを
\emph{接続している}といい, この接続状態を基にネットワーク全体の構造が形成される.
そして, このネットワーク構造を\emph{ネットワークトポロジー}という. 
VANETでは通信するノードを車両と想定しているため, 
移動速度の速いノードが動的にネットワークを形成する. 
そのため, 接続の頻繁な確立と切断が発生し, 
ネットワークトポロジーは急速に変化する. \\[1em]
\noindent\textbf{(4) 移動パターン}\\
\indent 車両の動きは道路や建造物などの物理的構造に従う.\\[1em]
\noindent\textbf{(5) 安全に関する情報のリアルタイム性}\\
\indent 交通事故や道路状況に関する情報を即座に共有するためには 
低遅延かつ信頼性の高い通信が求められる.\\

VANETには, 車両間通信や車両とインフラ間通信を実現するための
優れた技術としてのポテンシャルがある一方で, 
いくつかの課題も抱えている.
その中でも, セキュリティに関する課題はVANETを安全かつ信頼性の
高いシステムとして運用する上で最も重要な問題の一つとなっている
\cite{vanet-challenge,vanet-security}.\\

