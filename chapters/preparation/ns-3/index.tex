本研究では, シミュレーションツールとして\textbf{network simulator-3(ns-3)}
\cite{ns-3}を使用した. ns-3は, 離散事象ネットワークシミュレータであり, 
有線および無線通信プロトコルを含む多様なネットワークの
シミュレーションが可能なオープンソースソフトウェアである. 
ns-3のシステムは大きく分けて, シミュレーションの
実行を行う\textbf{ns-3 core}と, 実験の定義を行う
\textbf{simulation scenario}に分かれている. simulation scenarioには, 
特定の通信プロトコルやネットワーク技術 (Wi-FiやGPSRなど)を再現するための機能が
モジュール化されており, 複数のモジュールを組み合わることでシミュレーションを実現する. 
そのため, ユーザーは, 必要なモジュールを選択し, モジュール内で不足している部分を
適宜実装していくことで, 要件に応じたシミュレーションを実行することができる. 
開発言語はC++とPythonをサポートしている. \\
\indent ns-3では様々なコンテナと呼ばれるノードやネットワーク要素を効率的に
管理するためのデータ構造が存在し, それらを組み合わせて
プログラムを作成していく. 一般的には次の4つの主要なコンテナが使用される.
\begin{itemize}
  \item NodeContainer\\ 
  \indent ノードを管理するためのコンテナであり, 
  コンピュータやルータなど, 扱うデバイスが何であるかを示す. 
  \item DeviceContainer\\
  \indent 通信デバイスを管理するためのコンテナであり, 
  ノードがどのような通信機能をもつかを指定する. 
  \item InterfaceContainer\\
  \indent IPインターフェースを管理するためのコンテナであり, IPアドレスや
  サブネットマスクなどの情報を保持する.
  \item ApplicationContainer\\
  \indent アプリケーションレイヤのプログラムやサービスを管理するための
  コンテナであり, HTTPサーバやUDPアプリケーションなどが用意されている.
\end{itemize}
