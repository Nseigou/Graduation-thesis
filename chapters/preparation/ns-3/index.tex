本研究では, シミュレーションツールとして\textbf{network simulator-3(ns-3)}
\cite{ns-3}を使用した. ns-3は, 離散事象ネットワークシミュレータであり, 
有線および無線通信プロトコルを含む多様なネットワークの
シミュレーションが可能なオープンソースソフトウェアである. 
ns-3のシステムは大きく分けて, シミュレーションの
実行を行う\textbf{ns-3 core}と, 実験の定義を行う
\textbf{simulation scenario}に分かれている. simulation scenarioには, 
特定の通信プロトコルやネットワーク技術 (Wi-FiやGPSRなど)を再現するための機能が
モジュール化されており, 複数のモジュールを組み合わることでシミュレーションを実現する. 
そのため, ユーザーは, 必要なモジュールを選択し, モジュール内で不足している部分を
適宜実装していくことで, 要件に応じたシミュレーションを実行することができる. 
開発言語としてC++とPythonがサポートされている. \\
\indent ns-3では\textbf{コンテナ}と呼ばれる様々なノードやネットワーク要素を効率的に
管理するためのデータ構造が存在し, それらを組み合わせて
プログラムを作成していく. 一般的には, 表\ref{tab:container}に示す4つの主要なコンテナが使用される.
% \begin{itemize}
%   \item NodeContainer\\ 
%   \indent ノードを管理するためのコンテナであり, 
%   コンピュータやルータなど, 扱うデバイスが何であるかを示す. 
%   \item DeviceContainer\\
%   \indent 通信デバイスを管理するためのコンテナであり, 
%   ノードがどのような通信機能をもつかを指定する. 
%   \item InterfaceContainer\\
%   \indent IPインターフェースを管理するためのコンテナであり, IPアドレスや
%   サブネットマスクなどの情報を保持する.
%   \item ApplicationContainer\\
%   \indent アプリケーションレイヤのプログラムやサービスを管理するための
%   コンテナであり, HTTPサーバやUDPアプリケーションなどが用意されている.
% \end{itemize}
\begin{longtable}{cl}
  \caption{ns-3で使用される主要なコンテナ}
  \label{tab:container} \\
  \endfirsthead
  \hline
  \multicolumn{1}{c}{コンテナ名} & \multicolumn{1}{c}{説明} \\ \hline \hline
  NodeContainer & ノードを管理するためのコンテナであり, 
  コンピュータやルータなど,\\ 
  &扱うデバイスが何であるかを示す. \\
  InterfaceContainer & 通信デバイスを管理するためのコンテナであり, 
  ノードがどのような\\ 
  &通信機能をもつかを指定する.  \\
  DeviceContainer & IPインターフェースを管理するためのコンテナであり, IPアドレスや\\ 
  &サブネットマスクなどの情報を保持する. \\
  ApplicationContainer & アプリケーションレイヤのプログラムやサービスを管理するための\\
  &コンテナであり, HTTPサーバやUDPアプリケーションなどが\\
  &用意されている. \\ \hline
\end{longtable}
