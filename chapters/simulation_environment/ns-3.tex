\noindent{\Large\textbf{ns-3}}\\[1em]
\indent 本研究では, シミュレーションツールとして\textbf{network simulator-3(ns-3)}
\cite{ns-3}使用した. ns-3 は, 離散事象ネットワークシミュレータであり, 
有線および無線通信プロトコルを含む多様なネットワークの
シミュレーションが可能なオープンソースソフトウェアである. 
ns-3のシステムは大きく分けて, シミュレーションの
実行を行う\textbf{ns-3 core}と, 実験の定義を行う
\textbf{simulation scenario}に分かれている. ユーザーは, 自身の
シミュレーションの要求に自身のシミュレーションの要件に応じて不足している部分を
適宜実装し, シミュレーションを実行する. 開発言語はC++とPythonをサポートしているが, 
ns-3 coreはC++でのみ改変することができるため, 本研究ではC++でコーディングした.\\
\indent ns-3では様々なコンテナと呼ばれるノードやネットワーク要素を効率的に
管理するためのデータ構造が存在し, それらを組み合わせて
プログラムを作成していく. 一般的には次の4つの主要なコンテナが使用される.
\begin{itemize}
  \item NodeContainer\\ 
  \indent ノードを管理するためのコンテナであり, 
  コンピュータやルータなど, 扱うデバイスが何であるかを示す. 
  \item DeviceContainer\\
  \indent 通信デバイスを管理するためのコンテナであり, 
  ノードがどのような通信機能をもつかを指定する. 
  \item InterfaceContainer\\
  \indent IPインターフェースを管理するためのコンテナであり, IPアドレスや
  サブネットマスクなどの情報を保持する.
  \item ApplicationContainer\\
  \indent アプリケーションレイヤのプログラムやサービスを管理するための
  コンテナであり, HTTPサーバやUDPアプリケーションなどが用意されている.
\end{itemize}

ns-3はLinux環境での動作を前提に開発されている. そのため, 
本研究では仮想環境VMwareにUbuntuをインストールし, その上で
ns-3を動作させた. 現時点での ns-3 の最新バージョンは2024年10月9日
リリースのns-3.43となっているが, 本研究では, 先行研究
\cite{shinato}と互換性のあるns-3.26を使用した. \\[1em]


 
