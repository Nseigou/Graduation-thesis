この章では, 本研究で行ったシミュレーション環境について
述べる. シミュレーションのパラメータは表4.1の通りである.
\setlength{\tabcolsep}{30pt}
\begin{longtable}{ll}
  \caption{Ed25519のパラメータ}
  \endfirsthead
  \hline
  シミュレーションツール & ns-3.26 \\
  通信規格 & IEEE 802.11p \\
  通信プロトコル & UDP \\
  パケットサイズ & 1024 [byte] \\
  パケット送信間隔 & 1.0 [s] \\
  送信電力 & 17.026 [dBm] \\
  電力検出閾値 & -96.0 [dBm] \\
  電波伝搬減衰モデル & 対数距離電波伝搬減衰モデル \\
  遅延モデル & 定常速度伝搬モデル \\
  電波伝搬範囲 & 約300 [m] \\
  ノード数 & 74(実験1,2), 37/74/112/148/185(実験3) \\
  シミュレーション時間 & 300 [s] \\
  ルーティングプロトコル & GPSR \\
  デジタル署名 & DSA, ECDSA, EdDSA(Ed25519) \\ \hline
\end{longtable}
\vspace{3em}

{\LARGE\textbf{ns-3}}\\[1em]
\indent 本研究では, シミュレーションツールとして\textbf{network simulator-3(ns-3)}
\cite{ns-3}使用した. ns-3 は, 離散事象ネットワークシミュレータであり, 
有線および無線通信プロトコルを含む多様なネットワークの
シミュレーションが可能なオープンソースソフトウェアである. 
ns-3のシステムは大きく分けて, シミュレーションの
実行を行うns-3 coreと, 実験の定義を行うsimulation scenarioに
分かれている. ユーザーは, 自身のシミュレーションの要求に対する
空白部分を埋める形でコーディングし, シミュレーションを実行する. 
開発言語はC++とPythonをサポートしているが, ns-3 coreはC++でのみ
改変することができるため, 本研究ではC++でコーディングした.\\
\indent ns-3では様々なコンテナが存在し, それらを組み合わせて
プログラムを作成していく. 一般的には次の4つの主要なコンテナが使用される.
\begin{itemize}
  \item NodeContainer\\ 
  \indent ノードを管理するためのコンテナであり, 
  コンピュータやルータなど, 扱うデバイスが何であるかを示している. 
  \item DeviceContainer\\
  \indent 通信デバイスを管理するためのコンテナであり, 
  ノードがどのような通信機能をもつかを指定する. 
  \item InterfaceContainer\\
  \indent IPインターフェースを管理するためのコンテナであり, IPアドレスや
  サブネットマスクなどの情報を保持する.
  \item ApplicationContainer\\
  \indent アプリケーションレイヤのプログラムやサービスを管理するための
  コンテナであり, HTTPサーバやUDPアプリケーションなどが用意されている.
\end{itemize}

ns-3はLinux環境での動作を前提に開発されている. そのため, 
本研究では仮想環境VMwareにUbuntuをインストールし, その上で
ns-3を動作させた. 現時点での ns-3 の最新バージョンは2024年10月9日
リリースのns-3.43となっているが, 本研究では, 先行研究
\cite{shinato}と互換性のあるns-3.26を使用した. \\[1em]


 

\vspace{2em}
\noindent {\Large\textbf{通信規格 IEEE802.11p}}\\[1em]
\indent また, 本研究では通信規格としてIEEE802.11pを想定した. 
\textbf{IEEE802.11p}とは, 車車間通信(V2V)や路車間通信(V2I)を可能にするために
設計された無線通信規格であり, IEEEが定める802.11シリーズ(Wi-Fi規格)の
一部である. この規格は, 高度道路交通システム
(Intelligent Transportation System, ITS)の通信要件を満たすため, 
主に交通安全や効率化を目的としたアプリケーションに使用される.\\
\indent この規格の特徴を以下に示す. 
\begin{itemize}
  \item \textbf{OFDM(直行周波数分割多重方式)}\\
  \indent IEEE802.11pは, IEEE802.11aをもとに設計されており, 
  データ送信にOFDMを採用している. \textbf{OFDM}とは, サブキャリア
  (1つの通信チャネル(帯域幅)を細かく分割して得られる個々の周波数成分を持つ信号)を直交する形で並べて, 
  つまり, 互いに干渉しないように送信することにより, 周波数帯域を効率的に利用できる多重化技術である. 
  また, この方式は信号の反射による複数経路からの干渉(マルチパス干渉)に対して
  強い耐性を持ち, 高速移動環境下でも安定した通信を可能にする. さらに, 
  通信速度を表すデータレートは6Mbpsから27Mbpsの範囲で柔軟に設定できるため, 
  さまざまな通信条件に適応可能である. 
  \item \textbf{周波数帯域}\\
  \indent IEEE802.11pでは, 通信に使用される周波数帯域として
  5.850GHz~5.925GHz(5.9GHz帯)がITS専用として割り当てられている. 
  チャネル構成としては, 標準のWi-Fiで用いられる
  20MHzのチャネル幅を半減し, 10MHzのチャネル幅を採用している. 
  この10MHz単位の幅で7つのチャネルが定義されており, 
  各チャネルは10MHz間隔で配置される. 
  この中でも, 安全通信用として制御チャネル(CCH)とサービスチャネル(SCH)
  という, 2つの特別なチャネルが確保されている. 
  CCHは事故発生通知, 赤信号の警告, 緊急車両の接近通知などの緊急通信, 
  SCHは道路状況, 渋滞情報, 駐車場の空き情報などの便利な情報を伝えるための通信
  をするチャネルとなっており,  これらを時間で切り替えながら通信を行っている. 
  この設計により, リアルタイム性が求められる交通安全アプリケーションに適した
  高い信頼性と効率性を実現している. 
\end{itemize}

\vspace{2em}
\noindent{\Large\textbf{対数距離電波伝搬減衰モデル}}\\[1em]
\indent ns-3では通信環境をシミュレーションする際に, 
電波の伝搬特性を表すために, 様々な伝搬モデルが用意されている. 
本研究では, 都市, 郊外, 屋内といった様々な環境に適用できる
対数距離電波伝搬減衰モデルを使用した.\\ 
\indent 対数減衰モデルの定義を式(4.1)に示す. ここで, $d$は
送信機と受信機間の実際の距離, $d_0$は参照距離, $L$は距離$d$での
伝搬損失($dB$), $L_0$は参照距離$d_0$での伝搬損失, $n$は
環境依存のパケットロス指数である. \\
\begin{equation}
  L = L_0 + 10n\log_{10}\left(\frac{d}{d_0}\right)
\end{equation}
\vspace{1em}
\noindent{\Large\textbf{定常速度伝搬モデル}}\\
\indent 定常速度伝搬モデル(ConstantSpeedPropagationDelayModel)は, 
このモデルは電波が空間を伝搬する速度が一定であるという仮定に
基づいており, 伝搬遅延$\Delta t$は, 送信ノードと
受信ノード間の距離$d$と電波の伝搬速度$v$によって式(4.2)で定義される.
\begin{equation}
  \Delta t = \frac{d}{v}
\end{equation}
\vspace{2em}
\noindent{\Large\textbf{デジタル署名}}\\
\indent 本研究では, デジタル署名に1.3節で述べたDSAとECDSAに加え, 
2章で解説したEdDSA(Ed25519)を使用してシミュレーションを行い, 
デジタル署名方式による比較をした. \\ 
\indent DSAのセキュリティパラメータは以下の通りである. \\
\[
  (L,N)=(2048,256)
\]
\indent ECDSAでは, 楕円曲線にscep256klを使用した. 
その式とグラフ(図4.4)は以下の通りである. \\
\[
  y^2 = x^3 + 7 
\]

{\Large\textbf{図4.4}}

\indent EdDSAのパラメータは2.3節で述べたものを使用した.




\vspace{2em}
\noindent{\Large\textbf{移動モデル SUMO}}\\[1em]
\indent 本研究では, シミュレーション環境を実世界により近づけるために, 
Simulation of Urban Mobility (SUMO)\cite{sumo}とOpenStreetMapを利用して
岐阜駅近辺の交通データを取得し, 実際の車両の動きを再現するノードのモビリティモデルを
作成した. SUMOは都市環境における交通流動をシミュレーションするためのオープンソースソフ
トウェアである. 与えられた交通ネットワークから自動車, バス, 電車などで構成されてい
る交通流をシミュレーションすることができる. 
