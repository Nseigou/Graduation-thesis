\noindent{\Large\textbf{対数距離電波伝搬減衰モデル}}\\[1em]
\indent ns-3では通信環境をシミュレーションする際に, 
電波の伝搬特性を表すために, 様々な伝搬モデルが用意されている. 
本研究では, 都市, 郊外, 屋内といった様々な環境に適用できる
対数距離電波伝搬減衰モデルを使用した.\\ 
\indent 対数距離電波伝搬減衰モデルの定義を式\ref{log-distance}に示す. ここで, $d$は
送信機と受信機間の実際の距離, $d_0$は参照距離[m], $L$は距離$d$での
伝搬損失[$dB$], $L_0$は参照距離$d_0$での伝搬損失, $n$は
環境依存のパケットロス指数である. 参照距離とは, 電波の反射や回折などによる
減衰を考慮して決められる値である. \\
\begin{equation}\label{log-distance}
  L = L_0 + 10n\log_{10}\left(\frac{d}{d_0}\right)
\end{equation}