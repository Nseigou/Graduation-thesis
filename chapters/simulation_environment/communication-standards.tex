\noindent {\Large\textbf{通信規格 IEEE802.11p}}\\[1em]
\indent \textbf{IEEE802.11p}とは, 車車間通信(V2V)や
路車間通信(V2I)を可能にするために設計された無線通信規格であり, 
IEEEが定める802.11シリーズ(Wi-Fi規格)の一部である. 
この規格は, 高度道路交通システム
(Intelligent Transportation System, ITS)の通信要件を満たすため, 
主に交通安全や効率化を目的としたアプリケーションに使用される.\\
\indent この規格の特徴を以下に示す. 
\begin{itemize}
  \item \textbf{OFDM(直行周波数分割多重方式)}\\
  \indent IEEE802.11pは, IEEE802.11aを基に設計されており, 
  データ送信にOFDMを採用している. OFDMは, 狭帯域のサブキャリアを
  直交する形で並べて送信することで周波数帯域を効率的に利用できる
  多重化技術である. また, この方式は信号の反射による複数経路からの
  干渉(マルチパス干渉)に対して強い耐性を持ち, 高速移動環境下でも
  安定した通信を可能にする. さらに, データレートは6Mbpsから27Mbpsの
  範囲で柔軟に設定できるため, さまざまな通信条件に適応可能である. 
  \item \textbf{周波数帯域}\\
  \indent IEEE802.11pでは, 通信に使用される周波数帯域として
  5.850GHz~5.925GHz(5.9GHz帯)が専用のITSバンドとして
  割り当てられている. チャネル構成としては, 標準のWi-Fiで用いられる
  20MHzのチャネル幅を半減し, 10MHzのチャネル幅を採用している. 
  この10MHz単位の幅で7つのチャネルが定義されており, 
  各チャネルは中心周波数が10MHz間隔で配置される. 
  この中でも, 安全通信用として制御チャネルとサービスチャネル
  という2つの特別なチャネルが確保されている. この設計により, 
  リアルタイム性が求められる交通安全アプリケーションに適した
  高い信頼性と効率性を実現している. 
\end{itemize}
