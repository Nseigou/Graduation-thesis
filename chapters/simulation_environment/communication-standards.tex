\indent 本研究では, 通信規格としてIEEE802.11pを想定した. 
\textbf{IEEE802.11p}とは, 車車間通信(V2V)や車両インフラ間通信(V2I)を可能にするために
設計された無線通信規格であり, IEEEが定める802.11シリーズ(Wi-Fi規格)の
一部である. この規格は, 高度道路交通システム
(Intelligent Transportation System, ITS)の通信要件を満たすため, 
主に交通安全や効率化を目的としたアプリケーションに使用される.\\
\indent この規格の主な特徴を2つ挙げておく. 
\begin{itemize}
  \item \textbf{OFDM(直行周波数分割多重方式)}\\
  \indent IEEE802.11pは, IEEE802.11aをもとに設計されており, 
  データ送信にOFDMを採用している. 
  \textbf{OFDM (Orthogonal Frequency Division Multiplexing)}とは, サブキャリア
  (1つの通信チャネル(帯域幅)を細かく分割して得られる個々の周波数成分をもつ信号)を
  直交する形で並べることにより, つまり, 互いに干渉しないように送信することにより, 
  周波数帯域を効率的に利用できる多重化技術である. 
  この方式は信号の反射による複数経路からの干渉(マルチパス干渉)に対して
  強い耐性をもち, 高速移動環境下でも安定した通信を可能にする. さらに, 
  通信速度を表すデータレートは6Mbpsから27Mbpsの範囲で柔軟に設定できるため, 
  さまざまな通信条件に適応可能である. 
  \item \textbf{周波数帯域}\\
  \indent IEEE802.11pでは, 通信に使用される周波数帯域として
  5.850GHz~5.925GHz(5.9GHz帯)がITS専用として割り当てられている. 
  チャネル構成としては, 標準のWi-Fiで用いられる
  20MHzのチャネル幅を半減し, 10MHzのチャネル幅を採用している. 
  この10MHz単位の幅で7つのチャネルが定義されており, 
  各チャネルは10MHz間隔で配置される. 
  この中でも, 安全通信用として制御チャネル(CCH)とサービスチャネル(SCH)
  という, 2つの特別なチャネルが確保されている. 
  CCHは事故発生通知, 赤信号の警告, 緊急車両の接近通知などの緊急通信, 
  SCHは道路状況, 渋滞情報, 駐車場の空き情報などの便利な情報を伝えるための通信
  をするチャネルとなっており,  これらを時間で切り替えながら通信を行っている. 
  この設計により, リアルタイム性が求められる交通安全アプリケーションに適した
  高い信頼性と効率性を実現している. 
\end{itemize}
