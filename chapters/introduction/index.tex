1990年代後半から, 自動車の高度な通信技術を活用した車両アドホックネットワーク
 (Vehicular Ad Hoc Network, VANET)の研究が進められている. 
VANETは, 車車間 (Vehicle-to-Vehicle, V2V)や車両とインフラ間
 (Vehicle-to-Infrastructure, V2I)で通信を行い, 
リアルタイムな情報共有を通じて交通の安全性向上や渋滞緩和などの
交通の効率化を実現するための技術である. \\
\indent しかし, 無線通信を基盤とするVANETでは, データの改ざんや
なりすまし攻撃などのセキュリティリスクが存在し, 安全な通信を
確立するための対策が不可欠である. この問題に対処する手法の一つとして, 
デジタル署名を用いた認証技術がある. デジタル署名は, 送信者の
正当性を保証し, 通信の完全性を担保する役割を果たす. これまで, 
VANETのセキュリティを強化するために楕円曲線暗号 
(Elliptic Curve Cryptography, ECC)に基づくECDSA 
(Elliptic Curve Digital Signature Algorithm)が広く用いられてきた\cite{ravi}. 
しかし, ECDSAよりも高速で安全性の高いEdDSA 
(Edwards-curve Digital Signature Algorithm)が提案され, 
様々な分野での活用が進んでいる. \\
\indent 本研究では, VANETにおける代表的なルーティングプロトコルの
1つであるGPSR (Greedy Perimeter Stateless Routing)に, 
デジタル署名による認証を導入した階戸のセキュアな
V2Vアドホックネットワークルーティングプロトコル\cite{shinato}を研究の対象とする. 
このプロトコルは, 不正に働くノードを排除することでセキュリティを確保しているが, 
ECDSAを用いた認証機構を導入しているため, ネットワークにかかる負荷と
署名生成および検証にかかる処理計算が増えてしまうとい課題がある. 
この課題に対して本研究では, 署名にかかる計算の効率化を目指し,  
プロトコルで使用される署名方式をECDSAからEdDSAに置き換える. そうして実装したセキュアなGPSRで
シミュレーションを行い, その結果から, EdDSAがVANET環境においてどの程度の優位性を示すかを
明らかにする. \\
\indent 本論文は6章で構成される. 第1章では本研究の対象である
VANETやデジタル署名について解説する. 第2章では, 
本研究のベースとなる階戸が提案したセキュアなV2Vアドホックネットワーク
ルーティングプロトコル\cite{shinato}について解説する. 第3章では, 
プロトコルに導入するEdDSAについて説明する. 第4章で, 
シミュレーションを行う環境について述べた後, 第5章で
シミュレーション実験の内容とを示し, 実験結果を考察する. 
最後に, 第6章でEdDSAに関する実装評価について議論する. 

