近年, 自動車の高度な通信技術を活用した車両アドホックネットワーク
 (Vehicular Ad Hoc Network, VANET)の研究が進められている. 
VANETは, 車車間 (Vehicle-to-Vehicle, V2V)や車両とインフラ間
 (Vehicle-to-Infrastructure, V2I)で通信を行い, 
リアルタイムな情報共有を通じて交通の安全性向上や渋滞緩和などの
交通の効率化を実現するための技術である. \\
\indent しかし, 無線通信を基盤とするVANETでは, データの改ざんや
なりすまし攻撃などのセキュリティリスクが存在し, 安全な通信を
確立するための対策が不可欠である. この問題に対処する手法の一つとして, 
デジタル署名を用いた認証技術がある. デジタル署名は, 送信者の
正当性を保証し, 通信の完全性を担保する役割を果たす. これまで, 
VANETのセキュリティを強化するために楕円曲線暗号 
(Elliptic Curve Cryptography, ECC)に基づくECDSA 
(Elliptic Curve Digital Signature Algorithm)が広く用いられてきた. 
しかし, 近年ではより高速かつ安全性の高いEdDSA 
(Edwards-curve Digital Signature Algorithm)が提案され, 
様々な分野での活用が進んでいる. \\
\indent 本研究では, VANETにおける代表的なルーティングプロトコルの
1つであるGPSR (Greedy Perimeter Stateless Routing)に, 
デジタル署名による認証を導入した階戸\cite{shinato}のセキュアな
V2Vアドホックルーティングプロトコルを研究の対象とする. 
本研究では, このプロトコルの更なる効率化を目指す. そのために, 
プロトコルで使用される署名方式をECDSAをEdDSAに置き換えて
実装したものをシミュレーションし, その結果をもとに, 
EdDSAがVANET環境においてどの程度の優位性を示すかを明らかにする. \\
\indent 本論文は7章で構成される. 第1章では本研究の対象である
VANETやデジタル署名について解説する. 第2章では, 
本研究のベースとなる階戸\cite{shinato}が提案したセキュアなV2Vアドホック
ルーティングプロトコルについて解説する. 第3章では, 
プロトコルに導入するEdDSAについて説明する. 第4章では, 
EdDSAを導入する方法と評価基準について説明する. 第5章で, 
シミュレーションの環境について述べた後, 第6章で
シミュレーション実験の内容を示し, 実験結果を考察する. 
最後に第7章でEdDSAに関する実装評価について議論する. 

