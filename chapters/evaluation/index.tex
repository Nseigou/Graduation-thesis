この章では, 第3章で説明したEdDSAの性能と第5章で示した実験結果から, 
EdDSAの実装評価をする. はじめに, 第5章で示した実験結果の評価を
他の認証方式と比較しながらまとめ, その後に第3章の内容を含めながら
ECDSAとの総合的な比較評価を行う. 最後に, 本研究で行った実験結果から
想定されるEdDSAの利用シーンについて考察する.\\[1em]
\noindent {\large\textbf{実験結果のまとめ}} \\
\indent 第5章で行った実験の結果を以下にまとめる. 
\begin{enumerate}
  \item 実験1\\
  \indent EdDSAを用いたことで, 不正ノードを排除することができたが, 
  ECDSAとの差は見られず, セキュリティ性能は同等であった. 
  \item 実験2\\
  \indent 遅延時間とオーバーヘッドサイズにおいて, EdDSAは
  ECDSAとの差は見られず, ネットワークへの負荷は同程度であった.
  \item 実験3\\
  \indent EdDSAはECDSAよりも署名に関する計算効率が良いことから, 
  ネットワークの拡大に対するスケーラビリティが高いことがわかった.
\end{enumerate}

\noindent {\large\textbf{EdDSAとECDSAの比較評価}}\\
\indent 上記の実験結果のまとめから, EdDSAはECDSAよりも
計算効率が高いことがわかった. これは, 第3章で述べたEdDSAの設計上の
特長である高速な処理能力が実験環境においても十分に発揮されたことを
示している. また, 第3章で述べたセキュリティの堅牢性から, 
EdDSAは秘密鍵を特定しようとする攻撃者が存在する環境でECDSAよりも
高いセキュリティを提供できる. よって, V2Vアドホックネットワーク
において, EdDSAはECDSAよりも安全かつ効率的な認証方式として
利用できるといえる.\\

\noindent {\large\textbf{EdDSAの利用シーンについての考察}}\\
\indent シミュレーション実験からEdDSAはECDSAよりもスケーラビリティに優れていることがわかった. 
しかし, 本研究の実験環境では, 署名の生成と検証の時間が
パケットの送信間隔(1秒)に比べて非常に短い. そのため,
署名に関する処理が通信時間やスループットなどの通信品質に及ぼす影響はごくわずかであり,  
ECDSAとEdDSAの間で特徴的な差がでなかったと予想される. 
したがって, 次のような環境であれば署名の生成と検証の回数がより多くなり, 
ECDSAよりもEdDSAを用いるメリットがより顕著に表れるのではないかと考える. 
\begin{itemize}
  \item ノード数が非常に多い
  \item パケットの送信間隔が非常に短い
  \item 使えるリソースが限られている
\end{itemize}
さらに, 次のような環境であればEdDSAのセキュリティの堅牢さを
発揮すると考えられる. 
\begin{itemize}
  \item 秘密鍵を特定しようとする攻撃者が存在する
\end{itemize} 

\indent 上記のようなシチュエーションで実験を進めていくことで,
EdDSAがVANETに及ぼす影響をより具体的に明らかにできると考えられる. 






