本論文は筆者である永野が岐阜大学工学部電気電子・情報工学科情報コースに
在籍中の研究成果をまとめたものです. 本研究は多くの方々のご指導, 
ご協力のもと行われており, その方々の助力なくして, 
本研究は成立しませんでした. ここに深く感謝申し上げます. \\
\indent 筆者の指導教員である三嶋美和子教授および角田有助教には, 
研究の理論的な基盤の構築からその詳細な検討, 執筆活動にわたるまで, 
多くの助言と細やかなご指導をいただきました. 
ここに心から感謝の意を表します. \\
\indent また, 岐阜大学工学部フェロー原山美知子先生には, 
本研究における実践的な技術の整理や考察に関して, 的確な
アドバイスを数多くいただきました. 原山先生のご助言は, 
本研究の完成度を高める上で欠かせないものでした. 深く感謝申し上げます.\\ 
\indent 最後に, 本研究を遂行するにあたり, 三嶋研究室の皆様には
多大なご協力をいただきました. 苦楽を共にした同窓生である
大野氏, 野田氏, 加藤氏の3人に, 感謝いたします.\\[5em]

\begin{flushright}
  令和7年2月6日\\
  岐阜大学工学部電気電子・情報工学科情報コース\\
  永野 正剛
\end{flushright}
  