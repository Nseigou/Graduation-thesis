\documentclass[
  % LuaLaTeXを使う
  luatex,
  % 用紙サイズをA4にする
  paper=a4paper,
  % 欧文のフォントサイズを11ptにする
  fontsize=11pt,
  % レポート形式
  report,
  % 日本語組版処理の記述と矛盾する設定がある場合に通知
  jlreq_notes,
]{jlreq}

\usepackage{fontspec}
\usepackage[framemethod=tikz]{mdframed}
\usepackage{enumitem}
\usepackage{tocloft}  % 目次のカスタマイズ用パッケージ
\usepackage{geometry}  % ページレイアウトの調整
% jlreqsetupで色々設定できるようにする
\usepackage{jlreq-complements}
% 画像を扱う
\usepackage{graphicx}
% pdfのハイパーリンクを設定
\usepackage[hidelinks]{hyperref}
% 相対パスでファイルを読み込む
\usepackage{import}
% 枠をつける
\usepackage{ascmac}  
% 章やページの合計を扱う
\usepackage{totcount}
% svgを扱う
\usepackage{svg}
% 表の罫線を扱う
\usepackage{booktabs}
% 数式関連
\usepackage{amsmath,amssymb}
\usepackage{mathtools}
\usepackage[
  % mathtoolsと一部競合するため,警告を無視
  warnings-off={mathtools-colon,mathtools-overbracket}
]{unicode-math}
\unimathsetup{math-style=TeX,bold-style=TeX}
\setmainfont[Ligatures=TeX]{Latin Modern Roman}
\setsansfont[Ligatures=TeX]{Latin Modern Sans}
\setmonofont{Latin Modern Mono}
\setmathfont{Latin Modern Math}
% 目次の章タイトルを太字にする
\renewcommand{\cftchapfont}{\bfseries}  % 章タイトルを太字にする

% フォントを設定
% unicode-mathの後でないとフォントが変更されない
\usepackage[
  % 多ウェイト化を有効にする
  deluxe,
  % jlreqで使えるようにする
  jfm_yoko=jlreq,
  jfm_tate=jlreqv,
]{luatexja-preset}

% jlreqの設定
\jlreqsetup{
  % 参考文献の見出しの出力命令を設定
  thebibliography_heading={
    % 見出しを章にする
    \chapter*{\refname}
    \renewcommand{\cftsecfont}{\normalfont}  % 節タイトルを通常フォント
    % 目次に追加
    \addcontentsline{toc}{chapter}{\refname}
  },
}

\geometry{top=40mm, bottom=30mm, left=25mm, right=25mm}



\begin{document}

\begin{titlepage}
  \centering

  {\Large
    岐阜大学工学部 \\ [0.5em]
    電気電子・情報工学科 \\ [0.5em]
    令和6年度卒業論文 \\
  } 
  \vspace{8em} 

  {\huge
    セキュアなV2Vアドホックネットワーク \\
    ルーティングプロトコルのための \\
    EdDSA署名方式の評価 \\
  }
  \vspace{12em}

  \LARGE 三嶋研究室 \\ [1em]
  \Large 学籍番号:1213033107 \\ 
  \huge 永野 正剛 \\ [1em]
  \LARGE 指導教員:三嶋 美和子 教授 \\

  \vfill 
\end{titlepage}
% 最初のページはページ番号をローマ数字にする
\pagenumbering{roman}

% 目次を表示
\tableofcontents

\clearpage
% ページ番号を元に戻す
\pagenumbering{arabic}

\chapter*{はじめに}
\addcontentsline{toc}{chapter}{はじめに}  % 目次に追加
\chapter[   準備]{準備}
この章では, 本研究で使用する通信技術について説明する. 
1.1節では, VANETについて, 1.2節ではVANETで用いられる
ルーティングプロトコルであるGPSRについて述べる. 
1.3節では, VANETのセキュリティを担保するために使用する
デジタル署名について解説する.

\section{Vehicle Ad Hoc Network}
\textbf{VANET (Vehicle Ad Hoc Network)}とは, モバイル
アドホックネットワーク技術を車両間通信に応用したネットワークである
\cite{adhoc,vanet}. VANETは, 車両間の通信(Vehicle-to-Vehicle, 
\textbf{V2V}), および車両インフラ(路側機やドローン)間通信
(Vehicle-to-Infrastructure, \textbf{V2I})\cite{drone}で構成され, 
固定されたインフラに依存せず, ノード同士が自律的に
通信ネットワークを形成する. VANETでは, 車両間の距離が
無線通信の範囲を超えることが一般的であるため, 図\ref{fig:vanet}のように, 
データを送信元から宛先まで直接通信できない場合に, 
中継ノードを経由してデータを転送する. このような通信を
\textbf{マルチホップ通信}という.

\begin{figure}
  \centering
  \includegraphics[scale=0.6]{figures/vanet.png}
  \caption{マルチホップ通信}
  \label{fig:vanet}
\end{figure}

VANETには次の5つの特徴がある.\\[0.5em]
\noindent\textbf{(1) 十分な電力供給}\\
\indent VANETを利用する際, 車両は走行していることを前提とするため, 
スマートフォンのような電池駆動のモバイルデバイスに比べて電力の制約を
受けないと仮定する. したがって, 長時間の稼働や高い通信レートの実現が
可能であると仮定する. しかし, 高性能な通信モジュールやセンサーを
多数搭載する場合, 車両のエネルギー効率に影響を与える可能性が考えられるため, 
効率的なデバイス設計が求められる.\\[1em]
\noindent\textbf{(2) 位置情報の取得}\\
\indent 車両はGPSを搭載しているため, 自身の位置情報を取得できる.\\[1em]
\noindent\textbf{(3) ネットワークトポロジーの急速な変化}\\
\indent 無線通信では,ノード同士が直接的に通信可能であることを
\textbf{接続している}といい, この接続状態を基にネットワーク全体の構造が形成される. 
このネットワーク構造を\textbf{ネットワークトポロジー}という. 
VANETでは通信するノードを車両と想定しているため, 
移動速度の速いノードが動的にネットワークを形成する. 
そのため, 接続の頻繁な確立と切断が発生し, 
ネットワークトポロジーは急速に変化する. \\[1em]
\noindent\textbf{(4) 移動の制約}\\
\indent 車両の動きは道路や建造物などの物理的構造に従う.\\[1em]
\noindent\textbf{(5) 安全に関する情報のリアルタイム性}\\
\indent 交通事故や道路状況に関する情報を即座に共有するためには, 
低遅延かつ信頼性の高い通信が求められる.\\

VANETは, 車両間通信や車両インフラ間通信を実現するための
優れた技術としてのポテンシャルを有する一方で, 
いくつかの課題も抱えている.
その中でも, セキュリティに関する課題は, VANETを安全かつ信頼性の
高いシステムとして運用する上で最も重要な課題の1つとなっている
\cite{vanet-challenge,vanet-security}.\\


\section{Greedy Perimeter Stateless Routing}
\textbf{GPSR(Greedy Perimeter Stateless Routing)}\cite{gpsr}は, 
位置情報を利用してパケットを転送する位置ベースの
ルーティングプロトコルであり, VANETのような動的で高速に変化する
ネットワーク環境に適している.
GPSRでは, 自身の位置やIPアドレスなどの情報をのせたHelloパケットを
一定間隔で隣接ノードに送信する.
図1.2に示すように, それぞれのノードにはIPアドレスや隣接ノードの
位置などの情報が記載された隣接ノードテーブルをもち, 受信したHelloパケットの
情報を用いて隣接ノードテーブルを更新することにより周辺ノードの情報を把握する. 
取得した周辺ノードの位置情報を用いて, Greedy Forwarding と 
Perimeter Forwarding を組み合わせたルーティングプロトコルとなって
いる.\\

{\Huge 隣接ノードデーブルの図1.2を挿入}

\subsection{Greedy Forwarding}
Greedy ForwardingはGPSRの基本的なルーティング方式である.
図\ref{fig:greedy}のように, 送信ノード$S$は, 宛先ノード$D$の位置情報をもとに
自身の電波伝搬範囲内のノードから, $D$に最も近いノードを
ネクストホップとして選択する. ここで, 送信ノード$S$は宛先ノード$D$の
位置情報を事前に把握していることを前提とする. 点線で書かれた円は全ノードの
受信感度が等しい場合の送信ノード$S$の電波伝搬範囲を, 
破線は宛先ノードとの距離を表している.

\begin{figure}
  \centering
  \includegraphics[scale=0.7]{figures/greedy.png}
  \caption{Greedy Forwarding}
  \label{fig:greedy}
\end{figure}

Greedy Forwardingには\textbf{局所最大問題}が存在している. 
局所最大問題とは, 図\ref{fig:local}のように, 送信ノード$S$の電波伝搬範囲内に
宛先ノード$D$が存在しない, かつ, 送信ノード$S$が自身の
電波伝搬範囲内で宛先ノード$D$に最も近い場合, 選択できる
ネクストホップが存在しなくなるという問題である.

\begin{figure}
  \centering
  \includegraphics[scale=0.6]{figures/local.png}
  \caption{局所最大問題}
  \label{fig:local}
\end{figure}

\subsection{Perimeter Forwarding}
Greedy Forwardingで局所最大問題が発生した場合に, 
Perimeter Forwardingが使用される. 
図1.5のように, 送信ノードSを中心に
\textbf{Right Hand Rule} に則って反時計回りにノードを探索し, 
最初に発見したノードをネクストホップとして選択する方式である.

{\LARGE RightHandRulesの図1.5を挿入}
\section{デジタル署名}
デジタル署名書くよ

\chapter[   EdDSA]{EdDSA}
\section{データの変換}
EdDSAのアルゴリズム内では, 整数や点をオクテット列に変換するエンコードと
その逆変換であるデコードが行われる\cite{インフォーズ}.\\
 以下にEd25519で使用されるデータの変換について説明する.\\
なお, $b$は8の倍数とする. 
また, 16進数表記の数値は$0x$FFのようにその数値の前に
「$0x$」を付けることで表す.\\[1em]

\noindent{\large\textbf{ビット列}}\\
 8ビットからなるビット列
\[
b_0b_1b_2b_3b_4b_5b_6b_7
\]
を\emph{オクテット}と呼ぶ.
厳密には8ビット以外を指すこともある「バイト」の代わりに, 
必ず8ビットのことを指すものとして使われている語である.\\
ここで, 最下位ビットは$b_0$, 最上位ビットは$b_7$である.\\
\indent 例として, $0x12$は, 2進数では$00010010$であり, 
8ビットのビット列で表すと$01001000$となる.\\ 

\noindent{\large\textbf{リトルエンディアン形式}}\\
 リトルエンディアン形式とは, 数値をバイト単位で格納する際に, 
最下位バイト(数値の最小の値を持つバイト)を先頭に配置し, 
続けてその次に小さいバイトを配置していく方法を指す.
これは, 通常の十進法で右端から左へ数字を読むのと逆の順番で
データを並べることになる.\\
 例えば, 32ビットの $0x12345678$ をリトルエンディアン形式で
メモリに格納すると, 次のような順番でバイトが並ぶ:
\begin{itemize}
  \item 最下位バイト(1バイト目):$0x78$
  \item 2バイト目:$0x56$
  \item 3バイト目:$0x34$
  \item 最上位バイト(4バイト目):$0x12$
\end{itemize}
\noindent このようにして, $0x12345678$ はメモリ上で 
$0x78, 0x56, 0x34, 0x12$ の順番に格納される.\\[1em]

\noindent{\large\textbf{エンコードとデコード}}\\
\begin{enumerate}
  \item ENC$(s)$\\
   整数をオクテット列に変換し, データとして扱いやすくするための関数.\\
  処理:\\
   $b$ビットの整数$s$を入力として, $s$を リトルエンディアン形式にして
  $\tfrac{b}{8}$個のオクテットに変換して出力する.
  \item DEC$(t)$\\
   計算で使用できるよう, オクテット列を元の整数に戻す関数.\\
  処理:\\
   オクテット列$t$を入力として, 整数$s$に変換して出力する.
  つまり, $\mathrm{DEC}(t)=\mathrm{ENC}^{-1}(t)=s$である.
  \item ENCE$(A)$\\
   ツイストエドワーズ曲線上の点をオクテット列に変換し, 
  $x$座標の符号も付加して効率的に表現するための関数.\\
  処理:\\
   はじめに符号関数を
  \[
    \text{sign}(a) =
    \begin{cases}
    0 & \text{if } a \geq 0 \\
    1 & \text{if } a < 0
    \end{cases}
  \]
  とする.ここで$a$は整数である.\\
   ツイストエドワーズ曲線上の点A$(x,y)$の$y$を入力として, 
  ENC$(y)$によりオクテット列$y'=(y'_1,y'_2,...,y'_\frac{b}{8})$に変換し, 
  $y'_\frac{b}{8}$の最上位ビットにsign$(x)$を格納した
  オクテット列を出力する.
  \item DECE$(t)$\\
   オクテット列を再びツイストエドワーズ曲線上の点$(x, y)$に変換する関数.
  この変換の過程で符号や整合性のチェックを行い, 正しい点を復元する.\\
  処理:\\
   オクテット列$t$を入力として, 以下の手順でツイストエドワーズ曲線上の点$(x,y)$に
  変換して出力する.
  \begin{enumerate}
    \item[① ] $t$の最終オクテットの最上位ビットを$x$座標の符号として取り出し
    $x_0$に格納する.($x_0=0$ または, $x_0=1$とする.)
    \item[② ] $t$の最終オクテットの最上位ビットを0に設定する.
    \item[③ ] $y=$DEC$(t)$を計算し, $0\leq y<p$でないならばデコード失敗.
    \item[④ ] 以下の処理を行う.
    \begin{enumerate}
      \item $u=y^2-1$, $v=dy^2+1$として
      $x=uv^3(uv^7)^{\tfrac{p-5}{8}}\pmod p$を計算する.
      \item $vx^2 \neq \pm  u \pmod p$ならばデコード失敗とし, 処理を中断する.
      \item $vx^2=-u \pmod p$ならば, $x\leftarrow 2^{\tfrac{p-1}{4}}x$
    \end{enumerate}
    \item[⑤ ] $x=0$かつ$x_0=1$ならばデコード失敗とし, 処理を中断する.
    \item[⑥ ] $x_0$が$x \pmod 2$と異なるならば$x\leftarrow p-x$とする.
    \item[⑦ ] 点$(x,y)$を出力する.
  \end{enumerate}
\end{enumerate}


\section{EdDSA パラメータ}
EdDSAのパラメータは以下の通りである.なお, エドワーズ曲線を$E$とする.\\
\begin{table}[htbp]
  \centering
  \begin{tabular}{cp{10cm}}
    \hline
    \multicolumn{1}{c}{パラメータ} & \multicolumn{1}{c}{説明} \\ \hline \hline
    $p$ & 法となる素数. EdDSAは$\mathbb{F}_p$上の楕円曲線を使用する.\\
    $b$ & $b\geq 10$かつ$p<2^{b-1}$となる正整数. 公開鍵の長さを表す.\\
    H & ハッシュ関数. 2bビット長のハッシュ値を出力する. \\
    $a$ & $E$を決定するパラメータ. $\mathbb{F}_p$上の平方剰余. すなわち, $x^2\equiv d \pmod{p}$となる$x$が存在する.\\
    $d$ & $E$を決定するパラメータ. $a\neq d$. 非ゼロの非剰余. すなわち, $x^2\equiv d \mod{p}$となる$x$が存在しない.\\
    $c$ & $E$を決定するパラメータ. $2$または$3$. $2^{c}$はコファクタと呼ばれる.\\
    $L$ & $E$を決定するパラメータ. $2^{200}$より大きい奇素数で$E$の位数$#E=2^{c}l$となるような数であり, $B$の位数.\\
    $B$ & $E$上のベースポイント. $B\neq (0,1)$\\
    $n$ & $c\leq n < b$となる整数.\\ \hline
  \end{tabular}
  \caption{EdDSAのパラメータ}
\end{table}

本研究で使用するEd25519のパラメータは以下の通りである.\\
\begin{longtable}{cc}
  \caption{Ed25519のパラメータ}
  \endlastfoot
  \hline
  \multicolumn{1}{c}{パラメータ} & \multicolumn{1}{c}{値, または関数} \\ \hline \hline
  $p$ & $2^{255}-19$ \\
  $b$ & 256 \\
  H & SHA-512 \\
  $a$ & $-1$ \\
  $d$ & $-\frac{121665}{121666}$ \\
  $c$ & 3 \\
  $L$ & $2^{252} + 27742317777372353535851937790883648493$ \\
  $B$ & $(15112221349535400772501151409588531511454012693$ \\
  & $041857206046113283949847762202,$\\
  & $4631683569492647816942839400347516314130799386625622$ \\
  & $5615783033603165251855960)$ \\
  $n$ & 254 \\ \hline
\end{longtable}


\section{Ed25519}
EdDSAにはIETFのRFC8032で推奨される二つのパラメーターが存在する.
そのうちのひとつが本研究で使用するEd25519である.
現在、Ed25519 は EdDSA の最も一般的なインスタンスであり、
約 128 ビットのセキュリティを提供する Edwards Curve25519 
に基づいている.
\section{ECDSAとEdDSAの比較}
Ed25519はECDSAと比べ, 以下の点でセキュリティと処理時間が向上している.\\[1em]
{\large\textbf{セキュリティ}}\\[1em]
\noindent\textbf{決定論的な署名生成}\\
 Ed25519 は, 署名ごとにランダムな値を生成する代わりに, 
メッセージと秘密鍵をハッシュすることで 
ノンス(署名に使用するランダムな数値)を決定論的に生成する.
この方法により, 乱数生成の失敗による秘密鍵の漏洩リスクを回避している.
一方, ECDSA では, 乱数が予測可能または重複した場合, 秘密鍵が簡単に
推測されるリスクがあり, これがセキュリティの大きな弱点となる.\\[1em]
\noindent{\large\textbf{処理時間}}\\[1em]
 Ed25519 は, ねじれたエドワーズ曲線を使用しており, 
この形式の曲線は楕円曲線演算を効率的に実行できる特性を持っている.
特に, 加算と倍加の操作が簡素化され, 計算ステップが少なく済むため, 
ソフトウェアでの実装が高速になる.\\

\noindent\textbf{乱数生成の回避} \leavevmode\\
 EdDSA では乱数を鍵生成でのみ使用している.この乱数は秘密鍵に直接影響を
与えるため,暗号的に安全である必要がある.一方で,署名生成,署名検証では
乱数を使用しない.一般的に,暗号的に安全な乱数は生成の時間がランダムで
処理コストが高い.EdDSA では署名生成,署名検証を一定時間で行い,
乱数を使用するアルゴリズムに比べて高速に処理することができる.\\

\noindent\textbf{乗法逆元の不要} \leavevmode\\
 Ed25519では拡張した射影座標系(拡張ツイストエドワーズ座標)で
計算することがRFC8032で推奨されている.
この座標系の特性により, 処理コストの高い乗法逆元の処理が不要となり
処理が単純化され, 高速化しやすい.


\chapter[   提案手法]{提案手法}
位置情報を活用したルーティングプロトコルであるGPSRが正しく
機能するためには, 同一アドホックネットワーク内の
参加者が相互に正しい位置情報を送信する必要がある. 
しかし, ネットワーク内に不正なノードが存在する場合, 
ルーティングが妨害されかねない. この章では, 想定される 
不正ノードを示した後, その対策について述べる. \\

\noindent {\large\textbf{不正ノード}}\\[0.5em]
\noindent \textbf{1. 位置情報詐称ノード}\\
\indent 位置情報詐称ノードとは, 通信データの窃取を目的に, 
自身の位置情報を偽装してルーティングを妨害する内部不正ノードである. 
内部不正ノードとは, ネットワーク内において意図的に不正行為を行う
ノードのことで, 外部から侵入した攻撃者や, 内部の
信頼されたノードが乗っ取られることで発生することが一般的である. 
図\ref{fig:position-liar}のように, 送信ノード$S$が宛先ノード$D$に送信するとする. 
このとき, 本来$S$は電波伝搬範囲内で$D$に最も近い$A$を中継ノードとして
選択するべきであるが, ノード$B$が位置情報を詐称して$B'$の位置にいると
偽装することで, $B$を選択し, 誤ったルーティングをしてしまう. 
このようにして, $B$は本来$D$が受けるべきデータを窃取し, 
情報伝達を妨害できる.

\begin{figure}
  \centering
  \includegraphics[scale=0.7]{figures/position-liar.png}
  \caption{位置情報詐称ノード}
  \label{fig:position-liar}
\end{figure}

\noindent \textbf{2. IPアドレス詐称ノード}\\
\indent IPアドレス詐称ノードとは, ネットワークへの不正アクセスを
目的とした外部不正ノードである. 外部不正ノードとは, ネットワークの
外部から侵入し, 不正な目的でネットワーク内のデータやリソースを
攻撃または利用しようとするノードのことである. 図\ref{fig:ip-liar}は,  
当該ネットワークの構成ノードを黒丸, 非構成ノードを赤丸で示している. 
赤丸で示した外部ノードがネットワーク内のIPアドレスを詐称して
ネットワーク参加者に送信した場合, その情報を受け取ったノードは
外部ノードを信頼し, 通信を行ってしまう. このように, 
IPアドレスを詐称することで外部ノードがネットワークに不正に
参加できてしまう. 

\begin{figure}
  \centering
  \includegraphics[scale=0.7]{figures/ip-liar.png}
  \caption{IPアドレス詐称ノード}
  \label{fig:ip-liar}
\end{figure}

\indent このような2種類の不正ノードが存在する場合, 正しいルーティングが
行われない. そこで, 本研究では, デジタル署名を用いてノードの情報が
正しいものであるかを検証し, なりすましや改ざんを排除するようにした. \\
\indent では, その仕組みを説明する. 図\ref{fig:signature-method}に示すように, 
ノード$A$からノード$B$にデータを送信する場合, $A$は認証局による
署名をデータに付与して$B$に送信する. 
$B$は受信したデータに対して署名の検証を行い, ネットワーク内の
正当なノードから正しいデータが送信されたかを確認する.
\newpage

\begin{figure}
  \centering
  \includegraphics[scale=0.7]{figures/signature-method.png}
  \caption{本研究におけるデジタル署名の使用方法}
  \label{fig:signature-method}
\end{figure}

\indent 本研究で導入される認証機構は以下の通りである.
\setlength{\tabcolsep}{30pt}
\begin{longtable}{cc}
  \caption{認証機構(署名者)と被署名データの対応}
  \endfirsthead
  \hline
  \multicolumn{1}{c}{認証機構(署名者)} & \multicolumn{1}{c}{被署名データ} \\ \hline \hline
  DHCPサーバ & IPアドレス \\
  GPSまたは位置情報が正しいと証明できる機関 & 位置情報 \\ \hline
\end{longtable}
\vspace{2em}
\indent これらの認証機構による認証方法を次の手順で導入する(図\ref{fig:introduce}).
\begin{enumerate}
  \item 各ノードがDHCPサーバからIPアドレスを取得する際, 
  DHCPサーバから署名をもらう.
  \item 各ノードがGPSから位置情報を取得する際, GPSまたは
  位置情報が正しいと証明できる機関から署名をもらう.
  \item 各ノードが隣接ノードとHelloパケットを交換する際, 
  IPアドレスと位置情報の署名を一緒に送信する. 
  \item Helloパケットを受信したノードは2つの署名を検証し, 
  どちらの署名も検証が成功した場合のみ, 隣接ノードテーブルを
  更新する. どちらか一方でも署名検証に失敗した場合, 
  そのHelloパケットを破棄し, 隣接ノードテーブルの更新を行わない.
\end{enumerate}

\begin{figure}[h]
  \centering
  \includegraphics[scale=0.8]{figures/introduce.png}
  \caption{デジタル署名を用いた認証機構}
  \label{fig:introduce}
\end{figure}



\chapter[   シミュレーション環境]{シミュレーション環境}
この章では, 本研究で使用したシミュレーション環境について
述べる. シミュレーション環境と使用されるパラメータを与えた後, 
4.1節で, 通信規格IEEE 802.11pについて説明する. 4.2節では, 
VANET環境を再現するために使用した電波伝搬モデルについて述べる. 
4.3節では, 車の動きを再現するノードを作成するために使用した
移動モデルについて説明する.\\
\indent シミュレーションのパラメータは表\ref{tab:simulation_parameter}の通りである. 
\setlength{\tabcolsep}{30pt}
\begin{longtable}{ll}
  \caption{シミュレーション環境とパラメータ}
  \label{tab:simulation_parameter}
  \endfirsthead
  \hline
  ホストOS & Windows 10\\
  ゲストOS & Ubuntu 16.02\_LTS\\
  シミュレーションツール & ns-3.26 \\
  通信規格 & IEEE 802.11p \\
  通信プロトコル & UDP \\
  パケットサイズ & 1024 [byte] \\
  パケット送信間隔 & 1.0 [s] \\
  送信電力 & 17.026 [dBm] \\
  電力検出閾値 & -96.0 [dBm] \\
  電波伝搬減衰モデル & 対数距離電波伝搬減衰モデル \\
  遅延モデル & 定常速度伝搬モデル \\
  電波伝搬範囲 & 約300 [m] \\
  ノード数 & 74(実験1, 2), 37/74/112/148/185(実験3) \\
  シミュレーション時間 & 300 [s] \\
  ルーティングプロトコル & GPSR \\
  デジタル署名 & ECDSA, EdDSA \\ \hline
\end{longtable}
\vspace{3em}

\noindent{\Large\textbf{ns-3}}\\[1em]
\indent 本研究では, シミュレーションツールとして\textbf{network simulator-3(ns-3)}
\cite{ns-3}使用した. ns-3 は, 離散事象ネットワークシミュレータであり, 
有線および無線通信プロトコルを含む多様なネットワークの
シミュレーションが可能なオープンソースソフトウェアである. 
ns-3のシステムは大きく分けて, シミュレーションの
実行を行うns-3 coreと, 実験の定義を行うsimulation scenarioに
分かれている. ユーザーは, 自身のシミュレーションの要求に対する
空白部分を埋める形でコーディングし, シミュレーションを実行する. 
開発言語はC++とPythonをサポートしているが, ns-3 coreはC++でのみ
改変することができるため, 本研究ではC++でコーディングした.\\
\indent ns-3では様々なコンテナが存在し, それらを組み合わせて
プログラムを作成していく. 一般的には次の4つの主要なコンテナが使用される.
\begin{itemize}
  \item NodeContainer\\ 
  \indent ノードを管理するためのコンテナであり, 
  コンピュータやルータなど, 扱うデバイスが何であるかを示している. 
  \item DeviceContainer\\
  \indent 通信デバイスを管理するためのコンテナであり, 
  ノードがどのような通信機能をもつかを指定する. 
  \item InterfaceContainer\\
  \indent IPインターフェースを管理するためのコンテナであり, IPアドレスや
  サブネットマスクなどの情報を保持する.
  \item ApplicationContainer\\
  \indent アプリケーションレイヤのプログラムやサービスを管理するための
  コンテナであり, HTTPサーバやUDPアプリケーションなどが用意されている.
\end{itemize}

ns-3はLinux環境での動作を前提に開発されている. そのため, 
本研究では仮想環境VMwareにUbuntuをインストールし, その上で
ns-3を動作させた. 現時点での ns-3 の最新バージョンは2024年10月9日
リリースのns-3.43となっているが, 本研究では, 先行研究
\cite{shinato}と互換性のあるns-3.26を使用した. \\[1em]


 

\section{通信規格 IEEE 802.11p}
\noindent {\LARGE\textbf{通信規格 IEEE802.11p}}\\[1em]
\indent \textbf{IEEE802.11p}とは, 車車間通信(V2V)や
路車間通信(V2I)を可能にするために設計された無線通信規格であり, 
IEEEが定める802.11シリーズ(Wi-Fi規格)の一部である. 
この規格は, 高度道路交通システム
(Intelligent Transportation System, ITS)の通信要件を満たすため, 
主に交通安全や効率化を目的としたアプリケーションに使用される.\\
\indent この規格の特徴を以下に示す. 
\begin{itemize}
  \item \textbf{OFDM(直行周波数分割多重方式)}\\
  \indent IEEE802.11pは, IEEE802.11aを基に設計されており, 
  データ送信にOFDMを採用している. OFDMは, 狭帯域のサブキャリアを
  直交する形で並べて送信することで周波数帯域を効率的に利用できる
  多重化技術である. また, この方式は信号の反射による複数経路からの
  干渉(マルチパス干渉)に対して強い耐性を持ち, 高速移動環境下でも
  安定した通信を可能にする. さらに, データレートは6Mbpsから27Mbpsの
  範囲で柔軟に設定できるため, さまざまな通信条件に適応可能である. 
  \item \textbf{周波数帯域}\\
  \indent IEEE802.11pでは, 通信に使用される周波数帯域として
  5.850GHz~5.925GHz(5.9GHz帯)が専用のITSバンドとして
  割り当てられている. チャネル構成としては, 標準のWi-Fiで用いられる
  20MHzのチャネル幅を半減し, 10MHzのチャネル幅を採用している. 
  この10MHz単位の幅で7つのチャネルが定義されており, 
  各チャネルは中心周波数が10MHz間隔で配置される. 
  この中でも, 安全通信用として制御チャネルとサービスチャネル
  という2つの特別なチャネルが確保されている. この設計により, 
  リアルタイム性が求められる交通安全アプリケーションに適した
  高い信頼性と効率性を実現している. 
\end{itemize}

\section{電波伝搬モデル}
\noindent{\Large\textbf{対数距離電波伝搬減衰モデル}}\\[1em]
\indent ns-3では通信環境をシミュレーションする際に, 
電波の伝搬特性を表すために, 様々な伝搬モデルが用意されている. 
本研究では, 都市, 郊外, 屋内といった様々な環境に適用できる
対数距離電波伝搬減衰モデルを使用した.\\ 
\indent 対数減衰モデルの定義を式\ref{log-distance}に示す. ここで, $d$は
送信機と受信機間の実際の距離, $d_0$は参照距離, $L$は距離$d$での
伝搬損失($dB$), $L_0$は参照距離$d_0$での伝搬損失, $n$は
環境依存のパケットロス指数である. \\
\begin{equation}\label{log-distance}
  L = L_0 + 10n\log_{10}\left(\frac{d}{d_0}\right)
\end{equation}
\indent 一般的な無線通信の伝搬遅延を表すモデルに
定常速度伝搬モデルがある. 
\textbf{定常速度伝搬モデル(Constant Speed Propagation Delay Model)}は, 
電波が空間を伝搬する速度が一定であるという仮定に
基づいており, 伝搬遅延$\Delta t$は, 送信ノードと
受信ノード間の距離$d$と電波の伝搬速度$v$によって以下の式で定義される.
\[
  \Delta t = \frac{d}{v}
\]
\section{移動モデル}
\noindent{\Large\textbf{移動モデル SUMO}}\\[1em]
{\LARGE\textbf{\textcolor{red}{まだ書いてない}}}\\

\chapter[   シミュレーション実験]{シミュレーション実験}
本研究では, 第4章で述べたシミュレーション環境を用いて3種類の
シミュレーション実験を行った. いずれの実験でも, 以下の4パターンを調べた. 
\begin{itemize}
  \item 認証機構を用いない場合
  \item DSAを用いて認証機構を追加した場合
  \item ECDSAを用いて認証機構を追加した場合
  \item EdDSAを用いて認証機構を追加した場合
\end{itemize}
\indent シミュレーション結果を評価するために, 
以下の4つの評価基準を用いる. 
\begin{enumerate}
  \item \textbf{スループット(TP)}\\
  \indent \textbf{スループット}とは, 単位時間あたりに正常に転送された
  データ量のことである. 以下に示す式で定義される. ここで, 
  $AllTxBytes$は合計転送バイト数, $TxTimes$は
  転送にかかった時間である. \\
  \[
    TP = \frac{AllTxBytes}{TxTimes}\text{[kbps]}
  \]
  \item \textbf{遅延時間(DT)}\\
  \indent \textbf{遅延時間}とは, パケットが送信されてから受信されるまでの
  時間のことである. 以下に示す式で定義される. ここで,
  $TxPacketTime$は送信ノードがパケットを送信した時間, 
  $RxPacketTime$は宛先ノードがパケットを受信した時間である. \\
  \[
    DT = RxPacketTime - TxPacketTime \text{[ms]}
  \]

  \item \textbf{パケット配送率(PDR)}\\
  \indent \textbf{パケット配送率}とは, 送信されたデータ量のうち
  損失せずに受信されたデータ量の割合のことである. 以下に示す式で定義される. 
  ここで, $AllTxPackets$は合計送信パケット数, $AllRxPackets$は
  合計受信バイト数である. \\
  \[
    PDR = \frac{AllRxPackets}{AllTxPackets} \times 100 \text{[\%]}
  \]

  \item \textbf{オーバーヘッドサイズ(OH)}\\
  \indent \textbf{オーバーヘッドサイズ}とは, データの送受信に付随して発生する
  余分なコストのことであり, ここではルーティングに使用される
  通信データ量を指す. 以下に示す式で定義される. ここで, 
  $AllTxKBytes$はHelloパケットを含めた全ノードの合計送信キロバイト数, 
  $TxKBytes$はデータパケットの合計送信キロバイト数である. \\
  \[
    OH = AllTxKBytes - TxKBytes \text{[KB]}
  \]
\end{enumerate}
\vspace{2em}

\indent 3種類の実験の概要を述べる.\\
\indent 実験1では, 不正ノードが存在する環境で4パターンのシミュレーションを行い, 
認証機構の有効性を調査する. 具体的には, パケット配送率(PDR)とスループットを
測定し, EdDSAによってセキュリティがどの程度維持されているかを評価する. 
実験2では, 実験1の結果をもとに, 認証機構が正しく機能していることを
確認したうえで, 署名方式の違いが通信品質にどう影響するのかについて調査する. 
具体的には, 平均遅延時間と平均パケット配送率を測定し, オーバーヘッドサイズを
測定し, EdDSAが他の認証方式と比べてどのような特徴を持つのかを評価する.
実験3では, シミュレーション実行時間と署名生成, 検証にかかる時間を計測し, 
EdDSAの処理効率について評価する. \\


% \indent 本研究の目的は, EdDSAの性能評価を行うことであるので, 
% EdDSAと従来に最も処理効率の良かったECDSAの署名生成, 署名検証に
% 関するシミュレーション結果についても評価を行う.

\section{実験1}
実験1では, 認証機構が正しく機能していることを確認するための実験を
行った. 適度に影響が出るよう, 2章で述べた2種類の不正ノードを3個ずつ用意し, 
全ノードの約8\% (6個のノード)を不正ノードに設定した. また, 結果のばらつきを
抑えつつ, 統計的に有意な評価を可能にするために, そのような環境で
250回シミュレーションを行い, パケット配送率 (PDR)とスループットを
調べた. \\
\indent 実験の結果は以下の通りである. \\

\newpage
\begin{figure}
  \centering
  \includegraphics[width=1\textwidth]{figures/exp1_pdr.png}
  \caption{不正ノードが存在する環境でのパケット配送率}
  \label{fig:exp1_pdr}
\end{figure}
\clearpage
\begin{figure}
  \centering
  \includegraphics[width=1\textwidth]{figures/exp1_throughput.png}
  \caption{不正ノードが存在する環境でのスループット}
  \label{fig:exp1_throughput}
\end{figure}


\indent 図\ref{fig:exp1_pdr}は, 実験1におけるシミュレーションパターンごとの
パケット配送率を示している. 認証機構なしの場合に48.2 \%であるのに対し, DSAでは88.15 \%, 
ECDSAでは86.62 \%, EdDSAでは84.48 \%と, 認証機構を追加したことでパケット配送率が
約30 \%向上した. \\
\indent 図\ref{fig:exp1_throughput}は, 実験1におけるシミュレーションパターンごとの
スループットを示している. 認証機構なしの場合に4.18kbpsであるのに対し, 
DSAでは7.64kbps, ECDSAでは7.51kbps,EdDSAでは7.33kbpsと, 認証機構を
追加することでパケット配送率同様, スループットも向上した. \\
\indent これらの結果は, 認証機構の追加により 
不正ノードが排除されたことで, 経路選択を行う際に正当なノードのみを
選択しており,  データの窃取(転送中止)を回避できたことを示している. 
しかし, EdDSAの結果をDSAとECDSAの結果と比較するとほとんど差がないため, 
3つの署名方式が不正ノードを排除することにおいて同等の性能をもつと考えられる. 




\section{実験2}
実験2では, 認証機構の追加がネットワークにどれだけの負荷を与えるのかを
調査するための実験を行った. 具体的には, 
不正ノードの存在しない環境で, 250回シミュレーションを行い, 
遅延時間とオーバーヘッドサイズを調べた. 
実験2の主なシミレーションパラメータは表\ref{tab:exp2-params}に示す通りである. 
\begin{longtable}{cc}
  \caption{実験2のシミュレーションパラメータ}
  \label{tab:exp2-params}
  \endfirsthead
  \hline
  シミュレーション時間 & 300[s] \\
  ノード数 & 74 \\
  送受信ノードのペア数 & 1 \\ 
  不正ノード & なし \\ \hline
\end{longtable}
\vspace{1em}
\indent 実験結果を図\ref{fig:exp2_delay}, 図\ref{fig:exp2_overhead}に示す. \\[-2.5em]
\begin{figure}
  \centering
  \includegraphics[width=1\textwidth]{figures/exp2_delay.png}
  \caption{不正ノードが存在しない環境での遅延時間}
  \label{fig:exp2_delay}
\end{figure}
\begin{figure}
  \centering
  \includegraphics[width=1\textwidth]{figures/exp2_overhead.png}
  \caption{オーバーヘッドサイズ}
  \label{fig:exp2_overhead}
\end{figure}
\FloatBarrier
\indent 図\ref{fig:exp2_delay}は, 実験2におけるシミュレーションパターンごとの
遅延時間を示している. 認証なしでは6.16ms, ECDSAでは5.55ms, EdDSAでは5.31msであった. 
VANETがマルチホップ通信であることから, 選択する経路ごとに遅延が異なり, シミュレーションごとに
ばらつきが生じると考えられる. したがって, それぞれのプロトコルで若干の差異が見られるが, 
3つのプロトコルに特徴的な差が見られないことから, 認証機構の有無は遅延時間に影響を
与えなかったことがわかる. \\
\indent 図\ref{fig:exp2_overhead}は, 実験2におけるシミュレーションパターンごとの
オーバーヘッドサイズを示している. 認証機構なしでは2694.3KBであったのに対し, 
ECDSAでは5384.2KB, EdDSAでは5364.5KBと, 認証機構の追加により
オーバーヘッドサイズが大幅に増加した. これは, Helloパケットの
データに署名が付与されていることが原因である. しかし, EdDSAとECDSAではほとんど差がなかったため, 
署名方式の違いによるネットワークへの負荷は変わらないことがわかる. 
これは, ECDSA と EdDSA の鍵長および署名長が概ね同じ長さであることに起因する結果である. 
なお, 本研究で使用したパラメータによるそれぞれの鍵長は, 
ECDSAとEdDSAともに256ビット, 署名長はECDSAで64バイトから72バイト, EdDSAで64バイトである. \\



\section{実験3}
\indent 実験3では, 認証機構の処理能力(計算効率)を評価するための実験を行った. 
不正ノードの存在しない環境でノード数を37, 74, 112, 148, 185に設定して
250回ずつシミュレーションを行い, それぞれの実行にかかった時間を調べた.\\
\indent 実験の結果は以下の通りである. \\

{\LARGE\textbf{図5.6}}\\
{\LARGE\textbf{表5.1}}\\
{\LARGE\textbf{表5.2}}\\

\indent 図5.6にはシミュレーションパターンごとのノード数による
実行時間の変化を近似してグラフ化したものを, 表5.1にはその具体的な
値を示した. さらに, 表5.2にはECDSAとEdDSAにおいて, 1回の署名作成と
1回の署名検証にかかった時間を示した. \\
\indent 図5.6, 表5.1より, EdDSA, ECDSA, DSAの順番で実行時間が
短いのがわかる. また, 表5.2に示したように, EdDSAがECDSAに比べて
署名生成, 署名検証にかかる時間が短いことから, EdDSAは
3つの署名方式の中で最も計算効率が良いということが確かめられた. 
さらに, ECDSAとEdDSAの計算効率の差はノード数が増加するほど, 
実行時間全体に与える影響が大きなっており, EdDSAが
よりスケーラビリティに優れていることを示唆している. 
\chapter[   EdDSAに関する実装評価まとめ]{EdDSAに関する実装評価まとめ}
この章では, 第3章で説明したEdDSAの性能と第5章で示した実験結果から, 
EdDSAの実装評価をする. はじめに, 第5章で示した実験結果の評価を
他の認証方式と比較しながらまとめ, その後に第3章の内容を含めながら
ECDSAとの総合的な比較評価を行う. 最後に, 本研究で行った実験結果から
想定されるEdDSAの利用シーンについて考察する.\\[1em]
\noindent {\large\textbf{実験結果のまとめ}} \\
\indent 第5章で行った実験の結果を以下にまとめる. 
\begin{enumerate}
  \item 実験1\\
  \indent EdDSAを用いたことで, 不正ノードを排除することができたが, 
  ECDSAとの差は見られず, セキュリティ性能は同等であった. 
  \item 実験2\\
  \indent 遅延時間とオーバーヘッドサイズにおいて, EdDSAは
  ECDSAとの差は見られず, ネットワークへの負荷は同程度であった.
  \item 実験3\\
  \indent EdDSAはECDSAよりも署名に関する計算効率が良いことから, 
  ネットワークの拡大に対するスケーラビリティが高いことがわかった.
\end{enumerate}

\noindent {\large\textbf{EdDSAとECDSAの比較評価}}\\
\indent 上記の実験結果のまとめから, EdDSAはECDSAよりも
計算効率が高いことがわかった. これは, 第3章で述べたEdDSAの設計上の
特長である高速な処理能力が実験環境においても十分に発揮されたことを
示している. また, 第3章で述べたセキュリティの堅牢性から, 
EdDSAは秘密鍵を特定しようとする攻撃者が存在する環境でECDSAよりも
高いセキュリティを提供できる. よって, V2Vアドホックネットワーク
において, EdDSAはECDSAよりも安全かつ効率的な認証方式として
利用できるといえる.\\

\noindent {\large\textbf{EdDSAの利用シーンについての考察}}\\
\indent シミュレーション実験からEdDSAはECDSAよりもスケーラビリティに優れていることがわかった. 
しかし, 本研究の実験環境では, 署名の生成と検証の時間が
パケットの送信間隔(1秒)に比べて非常に短い. そのため,
署名に関する処理が通信時間やスループットなどの通信品質に及ぼす影響はごくわずかであり,  
ECDSAとEdDSAの間で特徴的な差がでなかったと予想される. 
したがって, 次のような環境であれば署名の生成と検証の回数がより多くなり, 
ECDSAよりもEdDSAを用いるメリットがより顕著に表れるのではないかと考える. 
\begin{itemize}
  \item ノード数が非常に多い
  \item パケットの送信間隔が非常に短い
  \item 使えるリソースが限られている
\end{itemize}
さらに, 次のような環境であればEdDSAのセキュリティの堅牢さを
発揮すると考えられる. 
\begin{itemize}
  \item 秘密鍵を特定しようとする攻撃者が存在する
\end{itemize} 

\indent 上記のようなシチュエーションで実験を進めていくことで,
EdDSAがVANETに及ぼす影響をより具体的に明らかにできると考えられる. 







\chapter*{おわりに}
\addcontentsline{toc}{chapter}{おわりに}  % 目次に追加
本論文は7章で構成される. 第1章では, VANETやデジタル署名について
解説した. 第2章では, 階戸\cite{shinato}のプロトコルについて説明した. 
第3章では, 本研究の対象であるEdDSA(Ed25519)の特徴や
アルゴリズムについて解説した. 第4章では, 階戸のプロトコルを更に
効率化させるための改良方法と, それを評価する方法について述べた. 
第5章では, シミュレーション環境について述べ, 第6章で
シミュレーションによる実験を行い, その結果と評価を示した. 
第7章では, 第3章と第6章の内容を踏まえ, EdDSAの実装評価について議論した. \\
\indent 本研究では, EdDSAの高い計算効率および堅牢なセキュリティ性能が, 
認証機構を追加したGPSRによるV2V通信にどのような影響を与えるかを
検証した. その結果, EdDSAを使用した場合, 署名生成および
検証にかかる処理時間が他の署名方式(ECDSA, DSA)よりも短縮され, 
ネットワーク全体の計算負荷が軽減されることを明らかにした. 
また, EdDSAの性能を最大限活かすための利用シーンの考察すると, 
EdDSAはV2VアドホックネットワークやIoTが発展に伴い, 
今後ますます重要な役割を果たすと考えられる.\\
\indent 今後の課題としては, 送受信ノードの数やノードの動き, 
通信のリンク状況などの要件を変更してシミュレーションを行い, 
多様な実験環境での影響を調べ, EdDSAのVANETへの適用可能性の調査を進めるとともに, 
より効率的な認証方法の探求も進めていきたい.

% また, 階戸のプロトコルでは, 位置情報とIPアドレスの認証が独立しており, 各ノードが
% 両方の検証を行っていたが, それらを統合し, ノードの認証をより効率的に
% 行う方法の探究もしていきたい. 
\chapter*{謝辞}
\addcontentsline{toc}{chapter}{謝辞}  % 目次に追加
本論文は筆者である永野が岐阜大学工学部電気電子・情報工学科情報コースに
在籍中の研究成果をまとめたものです. 本研究は多くの方々のご指導, 
ご協力のもと行われており, その方々の助力なくして, 
本研究は成立しませんでした. ここに深く感謝申し上げます. \\
\indent 筆者の指導教員である三嶋美和子教授および角田有助教には, 
研究の理論的な基盤の構築からその詳細な検討, 執筆活動にわたるまで, 
多くの助言と細やかなご指導をいただきました. 
ここに心から感謝の意を表します. \\
\indent また, 岐阜大学工学部フェロー原山美知子先生には, 
本研究における実践的な技術の整理や考察に関して, 的確な
アドバイスを数多くいただきました. 原山先生のご助言は, 
本研究の完成度を高める上で欠かせないものでした. 深く感謝申し上げます.\\ 
\indent 最後に, 本研究を遂行するにあたり, 三嶋研究室の皆様には
多大なご協力をいただきました. 苦楽を共にした同窓生である
大野氏, 野田氏, 加藤氏の3人に, 感謝いたします.\\[5em]

\begin{flushright}
  令和7年2月6日\\
  岐阜大学工学部電気電子・情報工学科情報コース\\
  永野 正剛
\end{flushright}
  
\chapter*{参考文献}
\addcontentsline{toc}{chapter}{参考文献}  % 目次に追加
\bibliographystyle{junsrt}
\bibliography{references}

\end{document}